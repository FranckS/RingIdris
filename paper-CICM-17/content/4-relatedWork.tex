\section{Related work}
\label{sect:relatedWork}

Coq's ring solver~\cite{Coq2005}, like ours, is
implemented using proof-by-reflection techniques, but without the guarantees obtained with
our type-safe reflection mechanism, and without the correct-by-construction
approach: first, they define the normalisation of terms, then
they prove correctness of the normalisation with an external lemma:
$\forall\ (e1\ e2\ :\ \code{Expr}),\ beq_{Expr}\ (\code{norm}\ e1)\
(\code{norm}\ e2)\ =\ true\ \rightarrow\ \code{reify}\ e1\ \simeq\
\code{reify}\ e2$.
This needs a \code{reify} function, which we do not need.
Furthermore, Coq's prover deals
with rings and semi-rings (commutative or not), but not with any of the
intermediate structures (semi-group, monoid, group, etc).
However,
their implementation has better performances due to the use of sparse normal
form and more optimised algorithms.  Our automatic reflection was written with
Idris reflection mechanism which allows pattern matching on syntax, and
their automatic reflection is programmed in Ltac~\cite{DelahayeLTac}, a proof
dedicated and untyped meta-language for the writing of automations.
More recently, the Mtac extension~\cite{Ziliani13} provides a
typed language for implementating proof automation.

As well as the ring solver, Coq also provides the
Omega solver~\cite{Cregut04}, which solves a goal in Presburger arithmetic
(i.e. a universally quantified formula made of equations and inequations), and
a field~\cite{DelahayeField} decision procedure for real numbers, which plugs
to Coq's ring prover after simplification of the multiplicative inverses.
Agda's reflection
mechanism\footnote{\url{http://wiki.portal.chalmers.se/agda/pmwiki.php?n=ReferenceManual.Reflection}}
gives access to a representation of the current goal (that is,
the required type) at a particular point in a program.
This allows various proof automations to be done in
Agda~\cite{DBLP:conf/mpc/KokkeS15,Lindblad04}. 

Proofs by reflection has been intensively
studied~\cite{ChlipalaBook,Malecha14}, but without anything similar to
the type-safe reflection that we have presented here.
%
If we leave the ground of nice mathematical structures, one can decide
to work with arbitrary rewriting rules, but in the general case there isn't a
complete decision procedure for such systems, because there is usually no
normal form. This is where deduction modulo~\cite{Dowek03,DelahayeModulo} and
proof search heuristics start.
