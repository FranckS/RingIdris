% Franck's notes on correctness and completeness !
% last revision : 30 Sept 2015
\documentclass{llncs}
%
\usepackage{makeidx}  % allows for indexgeneration
\usepackage{amsmath}
\usepackage{amssymb}
\usepackage{graphicx}
\usepackage{listings}
\usepackage[literate]{idrislang} 
%\usepackage{multicol}


%
\begin{document}
%
\frontmatter          % for the preliminaries
%
\pagestyle{headings}  % switches on printing of running heads
%\addtocmark{Hamiltonian Mechanics} % additional mark in the TOC
%

\mainmatter              % start of the contributions
%
\title{Notes on correctness and completeness of the approach we've followed and comparison with Coq} 
%
\titlerunning{Notes on correctness and completeness}  % abbreviated title (for running head)
%                                     also used for the TOC unless
%                                     \toctitle is used
%
\author{Franck Slama}
%
\authorrunning{Franck Slama} % abbreviated author list (for running head)
%
%%%% list of authors for the TOC (use if author list has to be modified)
%%%\tocauthor{}
%
\institute{University of St Andrews, St Andrews, United Kingdom\\
\email{fs39@st-andrews.ac.uk}
%,\\ WWW home page:
%\texttt{http://users/\homedir iekeland/web/welcome.html}
}

\maketitle              % typeset the title of the contribution

\abstract
I compare briefly Coq's ring prover and the provers for different algebraic structures we've built for Idris. I just focus on the correctness and completeness in theses notes.

\section{Correctness}

We note $C$ the concrete type on which we aim to prove things like $a=b$ (ie, both $a$ and $b$ have type $C$).
\\
\\
There is a type for reflected terms of type $C$, noted $ExprC$.
There could be a specialised type for normalised reflected elements $ExprNC$, but in our implementation, we still use $ExprC$ for that, which means that we've got $ExprNC = ExprC$ (this is not the case in Coq's implementation).
\\
\\
Now, something important and new in our approach, is that $ExprC$ is indexed over $C$ (i.e., $ExprC$ has type $C\ \rightarrow\ Type$). This index represents the concrete term (of type $C$) which is encoded by this encoding.
For each element $e$ of type $ExprC\ c$, we will note $e_c$ to explicitly show the index of this term (which can entirely be computed by the type-checker).
\\
\\
In the formalism presented in [1], which describes Coq's implementation of the Ring tactic, they use a function $\phi_{C} : ExprC\ \rightarrow\ C$ to get back the concrete element.
They've got the same thing for the type of normalised polynomials, with the function $\phi_{NC} : ExprNC\ \rightarrow\ C$. 
In our formalism, we don't need such functions, but we could write them. This function would simply return the index. If we don't want to cheat by simply returning the index (which is not good in term of erasure, even if it doesn't really matter here), we can easily recompute this index, and show that $\forall (c:C) (e:ExprC\ c), \phi_C (e) = c$ which really means that this function has in fact just returned the index.
\\
\\
In their formalism, they also use a metaification function, also called the function of automatic reflection. They define it in LTac, and we define it with  by using $\%syntax$ in Idris, which allows to pattern match on syntax rather that on constructors. The idea of this function is simply to compute the reflected term (of type ExprC c) representing a concrete value $c$ of type $C$.
In their formalism, this function isn't typed. Is our formalism, where the abstract values are index over the concrete one, and thanks to the fact that $\%syntax$ allows to produce typed terms, we define it with this type : $\tau : (c:C)\ \rightarrow\ ExprC\ c$. In this sense, we can talk about type safe reflection : the term produced by the automatic reflection is guarantee to be a faithful representation of the concrete term we've given in input.
\\
\\
Now, what we want to do is to try to prove $a=b$. For that, we will compute the encoding of $a$ and $b$, namely, $\tau\ a$ and $\tau\ b$.  The first thing to notice here is that $\tau\ a$ has for index $a$, as described by the type of the function $\tau$. We can thus write $(\tau\ a)_a$ in order to emphasize on the fact that $\tau\ a$ has for index $a$. We've got the same for $(\tau\ b)$ that we will write $(\tau\ b)_b$.
\\
\\
Now, we're going to run the normalisation algorithm on both $(\tau\ a)_a$ and $(\tau\ b)_b$. The normalisation procedure has this profile: \\
$norm : \forall\ c,\ ExprC\ c\ \rightarrow\ (c'\ **\ (Expr\ c',\ c=c'))$. \\
That means that running the normalisation algorithm on $(\tau\ a)_a$ will produce a dependent pair composed of an element $(a':C)$ and a term of type $ExprC \ a'$, that we write $(norm((\tau\ a)_a))_{a'}$, which is a term indexed over the new index $a'$. This function also produces a proof of $a=a'$.
\\
\\
The same thing happens when we run the normalisation algorithm on $(\tau\ b)_b$ : that produces a value $(b':C)$, a term $(norm((\tau\ b)_b))_{b'}$ which has for index the $b'$ which has just been produced, and a proof that $b=b'$.
\\
\\
Now, we're going to compare syntactically $(norm((\tau\ a)_a))_{a'}$ and $(norm((\tau\ b)_b))_{b'}$. What is really important is that if they're syntactically the same, then they're forced to be indexed over the same index : if an element of $ExprC\ a'$ equals (syntactically!) en element of $ExprC\ b'$, then this implies \footnote{In fact, in our implementation, we don't even bother and this function of syntactical comparison directly produces the proof of equality of the indices if the two reflected terms are syntactically the same : it has for type $(ExprC\ a') \rightarrow (ExprC\ b')\ \rightarrow\ Maybe(a'=b')$ and not $(e1:ExprC\ a') \rightarrow (e2:ExprC\ b')\ \rightarrow\ Maybe(e1=e2)$. This is simply because we're not interested in the proof of equality between the reflected terms, so instead of producing it, and having later an external lemma $extLemma:forall (a'\ b':C)\ (e1:ExprC\ a')\ (e2:ExprC\ b'),\ (e1=e2)\ \rightarrow\ (a'=b')$, which says that if two reflected terms are equal, then their index are also equal, we simply directly produce the proof we need.} $a'=b'$ \footnote{Note that the default equality is heterogeneous in Idris}.
\\
\\
So, if they're syntactically exactly equivalent, then we've also got $a'=b'$. But since we already had $a=a'$ and $b=b'$, we can compose these three proofs to produce the produce a proof of the desired property $a=b$. 
\\
In fact, everything has been made possible thanks to the following property (which doesn't even need to be expressed because it is already in the type of the function $norm$) : \\
$\forall (x\ y\ :\ C),\ norm ((\tau(x))_x) = Y_y \rightarrow x = y$
\\
\\
In their approach, because they don't follow this kind of "type-safe reflection", they do need a correctness lemma, which says : \\
$\forall (e:ExprC),\ \phi_{NC}\ (norm\ e)\ =\ \phi_C\ e$. \\
So, what will happen is that they will compute $(norm\ \tau(a))$ and $(norm\ \tau(b))$, exactly like us. 
Then, like us, they will compare these two values syntactically. If they are equal, then they get
$(norm\ \tau(a)) = (norm\ \tau(b))$. By applying $\phi_{NC}$ to both sides of this equality, they get $(\phi_{NC} (norm\ \tau(a))) = \phi_{NC} ((norm\ \tau(b)))$. But since $\phi_{NC} (norm\ \tau(a)) = \phi_C\ (\tau(a))$ and $\phi_{NC} (norm\ \tau(b)) = \phi_C\ (\tau(b))$ (from their correctness lemma), they get : \\
$\phi_C (\tau(a)) = \phi_C (\tau(b))$. \\
Now, if the functions $\phi_C$ and $\tau$ have been correctly defined, we should have $\forall x, \phi_C (\tau(x)) = x$. The problem is that this lemma hasn't been proved, and can't even be expressed, simply because the function $\tau$ is defined by using LTac. Unlike the $\%syntax$ directive in Idris, which still produces "normal" (typed) functions on which we can reason, LTac intend to define untyped things, which lie outside the world of terms and types of Coq.
\\
However, that doesn't break the correctness of their approach, because the correctness is in fact ensured by Coq's typechecker. When you try to produce a proof of $a=b$, there's no risk to produce a proof of $\phi_C (\tau(a)) = \phi_C (\tau(b))$, which is indeed an equality proof on the type $C$. If the typechecker will be able to unify together the left hand sides $a$ and $\phi_C (\tau(a))$, and together the right hand sides $b$ and $\phi_C (\tau(b))$, then the system will consider that they have generated a valid proof of $a=b$. If not, then the type-system will shout.
\\
Obviously, this can affect the completeness of the approach. Let's imagine that the function $\tau$ (defined in Ltac), always return the encoding of the constant $O:C$. Let's also imagine that the function $\phi_C$ is well defined (ie, it pattern matches on the constructor of ExprC, and returns the underlying expressions of type $C$). Then, every time we will apply this tactic, it will produce $expr_0$ (reflecting $O$) for both $\tau(a)$ and $\tau(b)$. The normalisation will give $norm(expr_O)$ for both $norm(\tau(a))$ and $norm(\tau(b))$ (we don't care of what it actually computes), which are therefore equal : 
$norm(expr_O) = norm(expr_O)$. By applying $\phi_{NC}$ on both side we will get $\phi_{NC} (norm(expr_O)) = \phi_{NC} (norm(expr_O))$. By applying the correctness lemma 2 times, we get :
$\phi_C (expr_O) = \phi_C (expr_O)$. Since $\phi_C$ is supposed to be correct, $\phi_C$ applied on the encoding of $O$ should give back $O$. We've therefore produce a proof of $0=0$. That means that this equality is the only one we can prove in this system where $\tau$ has been defined to always return the same term. We've completely lost the completeness.
\\
\\
Here, we see clearly the big advantage of our type-safe reflection : we're absolutely sure that $\tau(x)$ is a faithful representation of $x$, because $\tau(x)$ has for type $\tau : (c:C)\ \rightarrow\ ExprC\ c$. And we don't need a function $\tau$ to get back the concrete values, as we already have them available at the type level !
\\
\\
Now, one could be tempted to think that having the property $\forall x, \phi_C (\tau(x)) = x$ (which is not hard to get) is enough for ensuring the completeness. In fact, that's not enough, but that's necessary. As shown before, without this property, the prover they've build is still correct but might be completely unusable because completely incomplete. But that's not enough.



\end{document}