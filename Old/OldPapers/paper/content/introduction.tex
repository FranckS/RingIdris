\section{Introduction}


A very common situation in formal certification is to need to prove that one term is equal to another within some theories. Of course, they are usually not syntactically equal -this case would be trivial-, but they are equal according to a set of properties. These properties are either obtained by implementation (the implementation of + for relatives numbers is symmetric for example), or by axioms if we are working with an abstract structure.
These proofs of equality are usually done by hand, and are made with a potentially long sequence of rewriting, using the available properties that we have. For example, if $x$ and $y$ are relatives numbers, and if we want to show that :
$(x + y) * z = (z * x) + (y * z)$, we can first use the symmetry of $*$, saying that
$\forall a b, a * b = b * a$.
With the use of this property, we can rewrite the left part into the term $z * (x + y)$.
If now we use the distributivity of $*$ over $+$, which says that :
$\forall a b c, a * (b+c) = a*b + a*c$,
we now obtain the term $x*z + y*z$ for the left side.
If we use again the symmetry of $*$ on the subterm $(x*z)$, the left side of the equality becomes : $z*x + y*z$, which is what we want.





Thus, this kind of proofs consists of a potentially long sequence of rewriting, every rewriting step using one property of the theory. Without some specific automation, this sequence of rewriting is done by the user of the proof assistant. This is time consuming, and a little change in the left or the right hand side of the equality can invalidate completely the proof. The re-usability of this kind of proofs is indeed very low, since they are performing rewritings for a very specific term. In this paper, we describe a certified implementation of an automatic solver for equalities on algebraic structures, for the Idris language. Idris is a relatively new purely functional and dependently typed programming language with full dependent types. Idris also supports tactic based proofs. 

For our goal, we focus on some specific theories, which are the following algebraic structures : magmas, semi-group, monoid, group, Abelian group, ring and commutative ring.
For these structures, this is effectively possible to decide if a given equality is true, and to produce the proof of the equality is appropriate. This result comes directly from the fact that there exist a normal form for every element of these structures. The general idea is therefore to normalize both side of the equality, and then to compare the resulting terms using the usual syntactical Leibniz's equality.
This approach was followed by [paper Coq ring] for implementing a ring solver for the Coq proof assistant.

Our contribution mainly consists of two parts :
	- We follow a "correct by construction" approach for implementing the reduction procedures, instead of implementing a normalization procedure, and proving afterwards that this function effectively computes a normal form. For achieving this goal, we are extensively using dependent types in order to capture the interesting properties that matters for assuring the correctness of the method. 
	- We try to be as generic as possible, and re-use as much as possible the code of normalization from the structures below in the upper structures. For example developing an expression by using the axiom of distributivity works exactly the same in a commutative ring and in a ring, and this part of the reduction should be factorisable.