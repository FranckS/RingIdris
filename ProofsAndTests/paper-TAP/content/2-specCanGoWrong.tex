\section{Naive and usual approaches to the adequacy problem}

\label{sect:naiveApproaches}


When confronted to this problem of adequacy between the intuitive notion and the formalised one, a first possibility is to formalise the notion multiple times, with different predicates, and to prove that they are equivalent.
With our example, that means that we need to find another formalisation $isSorted'$ of being sorted, and to prove the following lemma.
$pred\_equiv : \forall\ (T:Type)\ (l:List\ T), (isSorted\ l) \leftrightarrow (isSorted'\ l)$.
This approach aims at increasing the confidence in our formal definitions by assuming that if we've managed to define multiple times the same notion, then we've surely succeeded to define the notion we wanted. The biggest problem with this approach is to be able to find some alternative formalisations that are sufficiently different from the original one. Obviously, if the new normalisations are too similar to the original one (and in the worst case the new ones are just syntactical variants of the first one), then we won't gain any guarantee. The ideal would be to capture the same notion by using very different point of view, and we will show in section~\ref{sect:testingInside} an original approach for doing so.

\

In order to gain confidence in the formal specifications we write, another traditional approach is to test the predicate on some values. That consists of defining a few terms, usually by hand, for which we know if the predicate should hold or not, and to prove that the predicate effectively holds when it should, and that it doesn't when it shouldn't. For example, with the predicate $isSorted$ defined in section 1, we can prove that it holds on the list [3, 5, 7] that we know sorted according to our informal sense of what being sorted is.

\begin{lstlisting}
isSorted_test1 : isSorted natIsOrdered [3, 5, 7]
isSorted_test1 = 
  let p1 : (3 <= 5) = tryDec (lowerEqDec natIsOrdered 3 5) in
  let p2 : (5 <= 7) = tryDec (lowerEqDec natIsOrdered 5 7) in
    ConsSorted 3 5 [7] (ConsSorted 5 7 [] (SingletonIsSorted _ 7) p2) p1
\end{lstlisting}
This test is a \emph{test done by proof} : we show that the predicate holds on some specific value, here $[3,\ 5,\ 7]$, by doing the proof. We can go a step forward by removing the need of doing these specific proofs by hand, because in this case, the predicate isSorted can be decided : there exists an algorithm that produces a proof of $(isSorted\ l)$ if appropriate, or a proof of $(not\ (isSorted\ l))$ otherwise.

\begin{lstlisting}
decideIsSorted : (Tord : Order T) -> (l:List T) -> Dec(isSorted Tord l)
decideIsSorted Tord [] = Yes (NilIsSorted Tord)
decideIsSorted Tord [x] = Yes (SingletonIsSorted Tord x)
decideIsSorted Tord (h1::(h2::t)) with (lowerEqDec Tord h1 h2)
  decideIsSorted Tord (h1::(h2::t)) | (Yes h1_lower_h2) 
          with (decideIsSorted Tord (h2::t))
    decideIsSorted Tord (h1::(h2::t)) | (Yes h1_lower_h2) 
          | (Yes h2_tail_sorted) = Yes (ConsSorted h1 h2 t h2_tail_sorted h1_lower_h2)
    decideIsSorted Tord (h1::(h2::t)) | (Yes h1_lower_h2) 
          | (No h2_tail_not_sorted) = No [...]
  decideIsSorted Tord (h1::(h2::t)) | No h1_not_lower_h2 = No [...]
\end{lstlisting}
Now, in order to do \emph{tests by proof}, we can simply run the decision procedure.

\begin{lstlisting}
isSorted_test1' : Dec (isSorted natIsOrdered [3,5,7])
isSorted_test1' = decideIsSorted natIsOrdered [3,5,7] 
\end{lstlisting}
And if we ask to evaluate $isSorted\_test1'$, the system will answer $Yes$ and the automatically generated proof of $isSorted\ [3,\ 5,\ 7]$, which is what we want, and we therefore know that the predicate has passed this test. With this technique, we can run semi-automatically a few tests on the predicate $isSorted$. It is semi-automatic in the sense that we still have to define by hand some terms that we know sorted and unsorted but we can let the machine produce the proof that the predicate holds or not on these specific values. This isn't bad --and this is in fact all that is done usually, when it is actually done-- but we would like to have a stronger guarantee, and not only that the predicate will coincide with our intuitive notion on a couple of tested terms.





