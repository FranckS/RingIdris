\section{Conclusions and future work}


We've presented some new ideas and also reintroduced some known techniques that aim to gain confidence in a predicate inside a proof assistant. Our new technique is based on the automatic generation of terms that should have the desired property. Thus, our goal was to make sure that the predicate and a generator of terms coincide. This adequacy between the predicate and the generator helps to gain confidence in the predicate, because generating examples of something and giving a definition of the same thing are tasks which are sufficiently different that it is very unlikely that they could coincide by luck. The technique presented on section \ref{sect:testingInside} was based on a finite observation of the automatically generated terms processed by a decision procedure. We've also shown on an example how these tests can be completely automated. In section \ref{sect:aStepForward} we've presented ideas for going a step forward and replacing these tests on the predicate by some new proofs. These new proofs have the advantage of not requiring the predicate to be decidable. However, they imply that we have to trust these new specifications, which are this time about the predicate itself.

We haven't been able to find much work done in the direction of predicate testing in the environment of proof assistants, but we strongly believe that this aspect is crucial, as there is absolutely no point to prove the correctness of a function relatively to a bad specification. 

A natural extension of the work presented here is to make it more generic, and to study how proof assistants could help --by providing some primitive bricks-- to gain confidence in the formal definitions that we write. A possibility is to provide many robust and generic concepts, like being sorted, so that the user won't have to define his own notion. 