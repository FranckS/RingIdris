\section{Testing the predicate by automatic generation of terms}

\label{sect:testingInside}

The key idea that this paper wants to convey is that it is often easier to generate examples of a notion than it is to precisely define it. We can operate an interesting change of point of view by generating the set of terms that we precisely wanted to describe with the predicate. With our example of sorted lists, that means that we need to define the same notion of being sorted, but this time with a definition example-based. Writing a function that produces all the sorted lists might be a bit more difficult than just giving a few examples, but this complicated function will have the main advantage of being so different from the predicate that if the two notions agree, then we will have gain a great confidence on the predicate. We decide to use coinduction and the type Stream (a coinductive version of lists, potentially infinite) in order to generate --with what we call a generator-- all the sorted lists of size $n$.

$generateSortedList : (T:Type)\ \rightarrow\ (recEnu:RecEnum\ T) \rightarrow\ (Tord : Order\ T) \rightarrow\ (n:Nat) \rightarrow Stream\ (List\ T)$.

To do so, the type $T$ needs to be recursively enumerable, which means that there must exist a computational map $Nat\ \rightarrow\ Maybe\ c$ with the condition that this map is surjective, which means that any value of type $c$ should be hit at least once by the map : $map\_is\_surjective : (y:c)\ \rightarrow\ (x:Nat\ **\ (computableMap\ x\ =\ Just\ y))$.

We want to check that this function and the predicate coincide. Since the predicate $isSorted$ is decidable with $decideIsSorted$, we can automatically check whether the generated sorted lists of size $n$ are automatically sorted (according to $isSorted$) by running this decision procedure. But since there can be infinitely many generated sorted lists of size $n$ when $n > 0$, we will only check that the generator and the predicate coincide on a finite observation\footnote{A finite observation of a stream, also called approximation at rank $m$ of a stream, is a vector of size $m$ that has the same $m$ first elements than the stream} of the resulting stream. The key point is that this observation can be arbitrary big, and the bigger it is, the better the guarantee is about $isSorted$.

We show how the observation is made on an example. Let $T$ be a type that only contains three constant values $A$, $B$ and $C$. Since $T$ is finite, it is therefore obviously recursively enumerable. We define on it the strict order $A < B < C$.
Let's automatically generate the first $m$ sorted lists of $T$, of size $n$, by unfolding $m$ times the result of $generateSortedList$.


\begin{lstlisting}
testGenerator : (m:Nat) -> (n:Nat) -> Maybe(Vect m (List T))
testGenerator m n = 
	let x = generateSortedList T TisRecEnu TisOrdered n 
	    in unfold_n_times x m
\end{lstlisting}
We can ask for the first 8 sorted lists of size 4 by evaluating $testGenerator\ 8\ 4$:

\begin{lstlisting}
Just [[A, A, A, A], [A, A, A, B], [A, A, A, C], [A, A, B, B],
      [A, A, B, C], [A, A, C, C], [A, B, B, B], [A, B, B, C]] 
      : Maybe (Vect 8 (List T))
\end{lstlisting}
Now, instead of simply generating the first $m$ sorted lists, we run the decision procedure on all of these $m$ tests in order to know if the predicate and the generator agree on this portion. The result will be a vector $m$ booleans.

\begin{lstlisting}
testSorted : (m:Nat) -> (n:Nat) -> Vect m Bool
testSorted m n = 
  let x = generateSortedList T TisRecEnu TisOrdered n in
    let y = Smap (\l => let res = decideIsSorted TisOrdered l in
			    case res of
			      Yes _ => True
			      No _ => False) x in
	unfold_n_times_with_padding y m True
\end{lstlisting}
And we can inspect the result of running the first 8 tests of size 4 by evaluating $testSorted\ 8\ 4$.

\begin{lstlisting}
[True, True, True, True, True, True, True, True] : Vect 8 Bool
\end{lstlisting}


When we want to test $isSorted$ on a large number of tests, we might not want to inspect manually the result of each test. We can write a function $testSorted\_result : (m:Nat)\ \rightarrow\ (n:Nat)\ \rightarrow\ Bool$ that calls $testSorted\ m\ n$ and does the boolean $And$ on each element of the resulting vector. Now, we can for example test the predicate on the first 50 sorted lists of size 9 by running $testSorted\_result\ 50\ 9$ and if we do so we get the overall result $True$ which means that the predicate agrees with the generator on all these 50 tests.

However, if the predicate $isSorted$ has been incorrectly written, then the result of this test might inform us that there's something wrong with the formal specification. For example, if we've forgotten the third constructor $consSorted$ in the definition of $isSorted$, then the result of $(testSorted\ 8\ 4)$ will be $False$, which means that at least one of the produced list is not seen as sorted according to $isSorted$, and we will therefore know that this predicate  does not capture our intuitive notion of sortedness.

In order to go a step forward, we can decide to replace the tests on the predicate by proofs. Instead of testing the predicate on a finite subset of all the generated sorted lists as we just did, we can try to prove that any of the automatically generated sorted list is provably sorted according to $isSorted$.

\begin{lstlisting}
generated_implies_pred_holds : {T:Type} 
   -> (recEnu:RecEnum T) -> (Tord : Order T) -> (n:Nat) 
   -> (All (generateSortedList T recEnu Tord n) 
           (\l => isSorted Tord l))
\end{lstlisting}
\begin{proof}
By induction on $n$.
When $n$ is zero, there is only one sorted list generated, which is the empty list, and we know that the empty list is sorted thanks to the constructor $NilIsSorted$. When n is some successor $(S\ pn)$, we know by using recursively the lemma on the smaller value $pn$ that all sorted lists of size $pn$ are sorted according to $isSorted$. Since the $Stream$ of all sorted lists of size $(S\ pn)$ has been made from the $Stream$ of all sorted lists of size $pn$ by adding to all of them --on the head position-- an element lower or equal to their respective current heads, we know that the property has been preserved at the higher rank.
\qed
\end{proof}

This lemma has the advantage of not requiring the predicate to be decidable, whereas this was needed when we automatically tested the predicate on a finite observation. However, one could object that this lemma is itself built by using a predicate, $All$, and that we can't necessary trust blindly such a specification. The answer is that no guarantee is perfect, and all we can do is to add some guarantees, but there is necessarily always something to trust. Moreover, this new kind of specifications and proofs --about the predicate itself-- uses more primitive components like streams and the predicate $All$, and these components can be provided once and for all. If they are part of some standard library being used intensively, there is very low risk that they do not capture the desired semantic.


