
\documentclass[runningheads,a4paper]{llncs}

\usepackage{amssymb}
\setcounter{tocdepth}{3}
\usepackage{graphicx}

\usepackage{url}
  
  
\usepackage{listings} 

\lstset{
  basicstyle=\ttfamily,
  columns=fullflexible,
  keepspaces=true,
  mathescape
}

  
\newcommand{\keywords}[1]{\par\addvspace\baselineskip
\noindent\keywordname\enspace\ignorespaces#1}

\begin{document}

\mainmatter  % start of an individual contribution

% first the title is needed
\title{Short paper : Automatic predicate testing in formal certification}
\subtitle{You've only proven what you've said, not what you meant!}

% a short form should be given in case it is too long for the running head
%\titlerunning{Automatic predicate testing in formal certification}


\author{Franck Slama\\
         \email{fs39@st-andrews.ac.uk}}

\institute{University of St Andrews, United Kingdom}

%\toctitle{Lecture Notes in Computer Science}
%\tocauthor{Authors' Instructions}
\maketitle


\begin{abstract}
The use of formal methods and proof assistants helps to increase the confidence in critical software. However, a formal proof is only a guarantee relative to a formal specification, and not necessary about the real requirements. There is always a jump when going from an informal specification to a formal specification expressed in a logical theory. Thus, proving the correctness of a piece of software always makes the implicit assumption that there is adequacy between the formalised specification --written with logical statements and predicates-- and the real requirements --often written in english--. Unfortunately, a huge part of the complexity lies precisely in the specification itself, and it is far from obvious that the formal specification says exactly and completely what it should say. Why should we trust more these predicates than the code that we've first refused to trust blindly, leading to these proofs? We show in this paper that the proving activity has not replaced the testing activity but has only changed the object which requires to be tested. Instead of testing code, we now need to test predicates. We present recent ideas about how to conduct these tests inside the proof assistant on a few examples, and how to automate them as far as possible.

\keywords{Formal certification, predicate testing, proof assistant}
\end{abstract}


\section{Introduction}


One way to increase the confidence in the software we build is to formally prove its correctness using a proof assistant. Proofs assistants enable to write code, logical statements and proofs in the same language, and offer the guarantee that every proof will be automatically checked. Many of them are functional programming languages, like Coq~\cite{Bertot2004}, Idris~\cite{brady2013idris} and Agda~\cite{Norell2008}, and others, like the B-Method~\cite{Abrial1991} belong to the imperative paradigm. These different paradigms are internally supported by different logic. Systems like Coq, Idris and Agda are based on various higher order logics (CoC, a variant of ML and LUO respectively) and are realisations of the Curry-Howard correspondence, while the formal B method is based on Hoare logic. These different foundations lead to different philosophies and different ways to implement and verify a software, but all of them greatly increase the confidence of the produced software. However, these guarantees tend to be too often considered as perfect, when they are in fact far from it. Knuth was --certainly ironically-- saying ``Beware of bugs in the above code; I have only proved it correct, not tried it". The reality is precisely that a proof is not enough. When we prove the correctness of a function, we only gain the guarantee expressed by the proven lemma, and nothing more. 

Say we want to implement a formally verified sorting function for list of elements of type $T$, where $T$ is ordered by a relation $\leq$.
We can decide to define the sorting function with a ``weak" type, like $sort\ :\ List\ T\ \rightarrow\ List\ T$, and to use an external lemma to ensure the correctness of the function. Which property does this function has to respect? First, the output has to be sorted, so we need to define this notion of being sorted, here as an inductive predicate :

\begin{lstlisting}
data isSorted : {T:Type} -> (Order T) -> (List T) -> Type where
    NilIsSorted : (Tord : Order T) -> isSorted Tord []
    SingletonIsSorted : (Tord : Order T) -> (x:T) -> isSorted Tord [x]
    ConsSorted : {Tord : Order T} -> (h1:T) -> (h2:T) -> (t:List T) 
                 -> (isSorted Tord (h2::t)) -> (h1 $\leq$ h2) 
                 -> (isSorted Tord (h1::(h2::t)))
\end{lstlisting}
The first and second constructor of this predicate say that $[\ ]$ and $[x]$ are sorted according to any order, and for any $x$. The third one says that a list of two or more elements is sorted if $h1 \le h2$, and if the list deprived from its head is also sorted. In order to express that the result of $sort$ is sorted, we can prove the following lemma:
$sort\_correct : \forall\ (T:Type)\ (Tord:Order\ T)\ (l:List\ T),\ isSorted\ Tord\ (sort\ l)$. The problem with this specification is that it does not say anything about the content of the output. The function $sort$ could just return the empty list $[\ ]$ all the time, it would still be possible to prove this correctness lemma. Here, the problem is that the function is underspecified, and it is therefore possible to write a senseless implementation, which is unfortunately provably correct. Only a careful reader could realise that the lemma $sort\_correct$ forgets to mention that the input and output list should be in bijection, meaning that everything which was originally in the input list should still be in the output, and that nothing else should have been added.

Another bug in the specification could have been to simply forget the third constructor $ConsSorted$. But things more nasty can happen. Imagine that the third constructor of $isSorted$, called $consSorted$ would have been written with a typo, and that the condition $(h1 \leq h2) $ would have been incorrectly written as $(h1 \leq h1)$. Any list would be seen as ``sorted", just because of this single typo, and the algorithm could for example return its input unchanged. One could object that when doing the proof of correctness, we should realize that the proof is being done too easily, without having to use the essential property that the output is being built such as any element in the list is always lower or equal than its next element. The reality is quite different because many effort are going in the direction of proof automation, which aims to let the machine generate automatically the proof for some kind of goals. For example, Coq has already a Ring prover~\cite{coq2005} and many others automations, and Idris has been recently equipped with a hierarchy of provers for algebraics structures~\cite{Slama2016}. There are even extensions to languages, such as Ltac~\cite{DelahayeLTac} and Mtac~\cite{Ziliani13} that aim to help the automation of tactics. The problem is that the machine is never going to find a proof ``too easy", and will never report that something seems weird with the specification given by the user.

Thus, if we want to trust the proven software, we're now forced to believe that there is adequacy between the formal specification and the informal requirements. A switch has occurred. We used to have to trust code, but we now have to trust logical statements and predicates. But when the specification is too often as complicated as the code, why should we blindly believe in it, when we've first refused to blindly trust the code? The primary aim of this paper is to raise awareness of the adequacy concern, and to see how heterogeneous approaches, that mix both proofs and tests, can help to go a step forward in the certification process, in the context of proof assistants based on type theory.
More precisely, we :
\begin{itemize}
	\item Show some basic approaches to the problem of underspecification (section~\ref{sect:naiveApproaches})
	\item Present a new way to test the predicate in the proof assistant, by automatically generating terms, and we completely automate these tests (section~\ref{sect:testingInside})
	\item Present how we can go a step forward by replacing these tests about the predicate by some proofs (section~\ref{sect:aStepForward})
\end{itemize}

We use Idris, a dependently typed programming language, but all the ideas that we present here can be applied to any proof assistant based on type theory. The running example that we use in this paper can be found online at \url{https://github.com/FranckS/}.




\section{Naive and usual approaches to the adequacy problem}

\label{sect:naiveApproaches}


When confronted to this problem of adequacy between the intuitive notion and the formalised one, a first possibility is to formalise the notion multiple times, with different predicates, and to prove that they are equivalent.
With our example, that means that we need to find another formalisation $isSorted'$ of being sorted, and to prove the following lemma.
$pred\_equiv : \forall\ (T:Type)\ (l:List\ T), (isSorted\ l) \leftrightarrow (isSorted'\ l)$.
This approach aims at increasing the confidence in our formal definitions by assuming that if we've managed to define multiple times the same notion, then we've surely succeeded to define the notion we wanted. The biggest problem with this approach is to be able to find some alternative formalisations that are sufficiently different from the original one. Obviously, if the new formalisations are too similar to the original one (and in the worst case the new ones are just syntactical variants of the first one), then we won't gain any guarantee. The ideal would be to capture the same notion by using very different points of view, and we will show in section~\ref{sect:testingInside} an original approach for doing so.

\

In order to gain confidence in the formal specifications we write, another traditional approach is to test the predicate on some values. That consists in defining a few terms, usually by hand, for which we know if the predicate should hold or not, and to prove that the predicate effectively holds when it should, and that it does not when it should not. For example, with the predicate $isSorted$ defined above, we can prove that it holds on the list [3, 5, 7] that we know sorted.

\begin{lstlisting}
isSorted_test1 : isSorted natIsOrdered [3, 5, 7]
isSorted_test1 = 
  let p1 : (3 <= 5) = tryDec (lowerEqDec natIsOrdered 3 5) in
  let p2 : (5 <= 7) = tryDec (lowerEqDec natIsOrdered 5 7) in
    ConsSorted 3 5 [7] 
      (ConsSorted 5 7 [] (SingletonIsSorted _ 7) p2) p1
\end{lstlisting}
This test is a \emph{test done by proof} : we show that the predicate holds on some specific value, here $[3,\ 5,\ 7]$, by doing the proof. We can go a step forward by removing the need of doing these specific proofs by hand, because in this case, the predicate isSorted can be decided : there exists an algorithm that produces a proof of $(isSorted\ l)$ if appropriate, or a proof of $(not\ (isSorted\ l))$ otherwise.

\begin{lstlisting}
decideIsSorted : (Tord : Order T) -> (l:List T) 
                 -> Dec(isSorted Tord l)
decideIsSorted Tord [] = Yes (NilIsSorted Tord)
decideIsSorted Tord [x] = Yes (SingletonIsSorted Tord x)
decideIsSorted Tord (h1::(h2::t)) with (lowerEqDec Tord h1 h2)
  | (Yes h1_lower_h2) with (decideIsSorted Tord (h2::t))
      | (Yes h2_tail_sorted) = Yes 
      		(ConsSorted h1 h2 t h2_tail_sorted h1_lower_h2) 
      | (No h2_tail_not_sorted) = No [...]
  | (No h1_not_lower_h2) = No [...]
\end{lstlisting}
Now, in order to do \emph{tests by proof}, we can simply run the decision procedure.

\begin{lstlisting}
isSorted_test1' : Dec (isSorted natIsOrdered [3,5,7])
isSorted_test1' = decideIsSorted natIsOrdered [3,5,7] 
\end{lstlisting}
And if we evaluate $isSorted\_test1'$, the system will answer $Yes$ and a proof of $isSorted\ [3,\ 5,\ 7]$, which means that the predicate has passed this test. With this technique, we can run semi-automatically a few tests on the predicate $isSorted$. It is semi-automatic in the sense that we still have to define by hand some terms that we know sorted and unsorted but we can let the machine produce the proof that the predicate holds or not on these specific values. This is not bad --and this is in fact all that is done usually, when it is actually done-- but we would like to have a stronger guarantee, and not only that the predicate will coincide with our intuitive notion on a couple of tested terms.








\section{Testing the predicate by automatic generation of terms}

\label{sect:testingInside}

The key idea that this paper wants to convey is that it is often easier to generate examples of a notion than it is to precisely define it. We can operate an interesting change of point of view by generating the set of terms that we precisely wanted to describe with the predicate. With our example of sorted lists, that means that we need to define the same notion of being sorted, but this time with a definition example-based. Writing a function that produces all the sorted lists might be a bit more difficult than just giving a few examples, but this complicated function will have the main advantage of being so different from the predicate that if the two notions agree, then we will have gain a great confidence on the predicate. We decide to use coinduction and the type Stream (a coinductive version of lists, potentially infinite) in order to generate --with what we call a generator-- all the sorted lists of size $n$.

$generateSortedList : (T:Type)\ \rightarrow\ (recEnu:RecEnum\ T) \rightarrow\ (Tord : Order\ T) \rightarrow\ (n:Nat) \rightarrow Stream\ (List\ T)$.

To do so, the type $T$ needs to be recursively enumerable, which means that there must exist a computational map $Nat\ \rightarrow\ Maybe\ c$ with the condition that this map is surjective, which means that any value of type $c$ should be hit at least once by the map : $map\_is\_surjective : (y:c)\ \rightarrow\ (x:Nat\ **\ (computableMap\ x\ =\ Just\ y))$.

We want to check that this function and the predicate coincide. Since the predicate $isSorted$ is decidable with $decideIsSorted$, we can automatically check whether the generated sorted lists of size $n$ are automatically sorted (according to $isSorted$) by running this decision procedure. But since there can be infinitely many generated sorted lists of size $n$ when $n > 0$, we will only check that the generator and the predicate coincide on a finite observation\footnote{A finite observation of a stream, also called approximation at rank $m$ of a stream, is a vector of size $m$ that has the same $m$ first elements than the stream} of the resulting stream. The key point is that this observation can be arbitrary big, and the bigger it is, the better the guarantee is about $isSorted$.

We show how the observation is made on an example. Let $T$ be a type that only contains three constant values $A$, $B$ and $C$. Since $T$ is finite, it is therefore obviously recursively enumerable. We define on it the strict order $A < B < C$.
Let's automatically generate the first $m$ sorted lists of $T$, of size $n$, by unfolding $m$ times the result of $generateSortedList$.


\begin{lstlisting}
testGenerator : (m:Nat) -> (n:Nat) -> Maybe(Vect m (List T))
testGenerator m n = 
	let x = generateSortedList T TisRecEnu TisOrdered n 
	    in unfold_n_times x m
\end{lstlisting}
We can ask for the first 8 sorted lists of size 4 by evaluating $testGenerator\ 8\ 4$:

\begin{lstlisting}
Just [[A, A, A, A], [A, A, A, B], [A, A, A, C], [A, A, B, B],
      [A, A, B, C], [A, A, C, C], [A, B, B, B], [A, B, B, C]] 
      : Maybe (Vect 8 (List T))
\end{lstlisting}
Now, instead of simply generating the first $m$ sorted lists, we run the decision procedure on all of these $m$ tests in order to know if the predicate and the generator agree on this portion. The result will be a vector $m$ booleans.

\begin{lstlisting}
testSorted : (m:Nat) -> (n:Nat) -> Vect m Bool
testSorted m n = 
  let x = generateSortedList T TisRecEnu TisOrdered n in
    let y = Smap (\l => let res = decideIsSorted TisOrdered l in
			    case res of
			      Yes _ => True
			      No _ => False) x in
	unfold_n_times_with_padding y m True
\end{lstlisting}
And we can inspect the result of running the first 8 tests of size 4 by evaluating $testSorted\ 8\ 4$.

\begin{lstlisting}
[True, True, True, True, True, True, True, True] : Vect 8 Bool
\end{lstlisting}


When we want to test $isSorted$ on a large number of tests, we might not want to inspect manually the result of each test. We can write a function $testSorted\_result : (m:Nat)\ \rightarrow\ (n:Nat)\ \rightarrow\ Bool$ that calls $testSorted\ m\ n$ and does the boolean $And$ on each element of the resulting vector. Now, we can for example test the predicate on the first 50 sorted lists of size 9 by running $testSorted\_result\ 50\ 9$ and if we do so we get the overall result $True$ which means that the predicate agrees with the generator on all these 50 tests.

However, if the predicate $isSorted$ has been incorrectly written, then the result of this test might inform us that there's something wrong with the formal specification. For example, if we've forgotten the third constructor $consSorted$ in the definition of $isSorted$, then the result of $(testSorted\ 8\ 4)$ will be $False$, which means that at least one of the produced list is not seen as sorted according to $isSorted$, and we will therefore know that this predicate  does not capture our intuitive notion of sortedness.

In order to go a step forward, we can decide to replace the tests on the predicate by proofs. Instead of testing the predicate on a finite subset of all the generated sorted lists as we just did, we can try to prove that any of the automatically generated sorted list is provably sorted according to $isSorted$.

\begin{lstlisting}
generated_implies_pred_holds : {T:Type} 
   -> (recEnu:RecEnum T) -> (Tord : Order T) -> (n:Nat) 
   -> (All (generateSortedList T recEnu Tord n) 
           (\l => isSorted Tord l))
\end{lstlisting}
\begin{proof}
By induction on $n$.
When $n$ is zero, there is only one sorted list generated, which is the empty list, and we know that the empty list is sorted thanks to the constructor $NilIsSorted$. When n is some successor $(S\ pn)$, we know by using recursively the lemma on the smaller value $pn$ that all sorted lists of size $pn$ are sorted according to $isSorted$. Since the $Stream$ of all sorted lists of size $(S\ pn)$ has been made from the $Stream$ of all sorted lists of size $pn$ by adding to all of them --on the head position-- an element lower or equal to their respective current heads, we know that the property has been preserved at the higher rank.
\qed
\end{proof}

This lemma has the advantage of not requiring the predicate to be decidable, whereas this was needed when we automatically tested the predicate on a finite observation. However, one could object that this lemma is itself built by using a predicate, $All$, and that we can't necessary trust blindly such a specification. The answer is that no guarantee is perfect, and all we can do is to add some guarantees, but there is necessarily always something to trust. Moreover, this new kind of specifications and proofs --about the predicate itself-- uses more primitive components like streams and the predicate $All$, and these components can be provided once and for all. If they are part of some standard library being used intensively, there is very low risk that they do not capture the desired semantic.





\section{Replacing tests on the predicate by proofs}

\label{sect:aStepForward}

In order to go a step forward, we can decide to replace the tests on the predicate by proofs. Instead of testing the predicate on a finite subset of all the generated sorted lists, as we did in the previous section, we can try to prove that any of the automatically generated sorted list is provably sorted according to $isSorted$.

\begin{lstlisting}
generated_implies_pred_holds : {T:Type} 
   -> (recEnu:RecEnum T) -> (Tord : Order T) -> (n:Nat) 
   -> (All (generateSortedList T recEnu Tord n) 
           (\l => isSorted Tord l))
\end{lstlisting}
\begin{proof}
By induction on $n$.
When $n$ is zero, there is only one sorted list generated, which is the empty list, and we know that the empty list is sorted thanks to the constructor $NilIsSorted$. When n is some successor $(S\ pn)$, we know by using recursively the lemma on the smaller value $pn$ that all sorted lists of size $pn$ are sorted according to $isSorted$. Since the $Stream$ of all sorted lists of size $(S\ pn)$ has been made from the $Stream$ of all sorted lists of size $pn$ by adding to all of them --on the head position-- an element lower or equal to the current heads, we know that the property has been preserved at the higher rank.
\qed
\end{proof}

This lemma has the advantage of not requiring the predicate to be decidable, when this was needed in the previous section where we automatically tested the predicate on a finite observation. However, one could object that this lemma is itself built by using a predicate, $All$, and that we can't necessary trust blindly such a specification. The answer is that no guarantee is perfect, and all we can do is to add more guarantee, but there is always something to trust. Moreover, this new kind of specifications and proofs --about the predicate itself-- use more primitive components like streams and the predicate $All$, and these components can be provided once and for all. If they are part of some standard library being used intensively, there is very low risk that they don't capture the desired semantic.

In the other direction, we can prove that when the predicate $isSorted$ holds on some list, then this list is one of the list produced by the generator. The proof is done by induction on the list. For a singleton $[x]$, the proof uses the fact that $T$ is recursively enumerable, which means that the map has hit any value, and thus in particular $x$. When there are two or more elements, we know that $isSorted$ holds on the smaller list deprived from its head by construction. We finish the proof by using the lemma recursively on this smaller list together with the fact that the head is lower than the next element and the fact that the map is surjective.\qed



\section{Conclusions and future work}


We've presented some new ideas and also reintroduced some known techniques that aim to gain confidence in a predicate inside a proof assistant. Our new technique is based on the automatic generation of terms that should have the desired property. This adequacy between the predicate and the generator helps to gain confidence in the predicate. The technique presented on section \ref{sect:testingInside} was based on a finite observation of the infinitely many generated terms processed by a decision procedure. We've also shown on an example how these tests can be completely automated. In section \ref{sect:aStepForward} we've presented ideas for going a step forward and replacing these tests on the predicate by some new proofs. These new proofs have the advantage of not requiring the predicate to be decidable. However, they imply that we have to trust these new specifications, which are this time about the predicate itself.

We haven't been able to find much work done in the direction of predicate testing in the environment of proof assistants, but we strongly believe that this aspect is crucial, as there is absolutely no point to prove the correctness of a function relatively to a bad specification. A natural extension of the work presented here is to make it more generic, and to study how proof assistants could provide some primitive bricks that could help to gain confidence in the formal definitions that we write. A possibility is to provide many robust and generic concepts, like being sorted, so that the user won't have to define his own properties. 



\bibliographystyle{splncs03}

\nocite{DeMillo1979}

\bibliography{bib-TAP}

\end{document}
