TO BE CONTINUED

Idris is a purely functional programming language fairly recent. Its main purpose is to bring formal certification into the scope of programming : instead of Coq or Isabelle, Idris is not intended to be a proof assistant, but as a real programming language usable by non-logicians. One of its more interesting features -and the one that enables to effectively manipulate logical statements and proofs in the language- is its full and effective support for dependent types. Especially, Idris has a particularly efficient dependent pattern matching, which is something mandatory for writing code correct by construction.

\subsection{Some common datatypes}

Maybe et Dec

\subsection{Dependent pattern maching}

Idris supports dependent pattern maching. Pattern maching a term can gives information about the shape of another term. For example, when pattern maching a $n:nat$ and a $v:Vect\ n\ T$ simultaneously, the case $Z$ for $n$ forces $v$ to be the empty vector $Nil$.
The following function $foo$ is therefore spotted as total by the Idris compiler.

\begin{code}[caption=A function known to be total thanks to dependent pattern maching, captionpos=b, label=lst1:haskell2]
foo : (n:Nat) -> (Vect n T) -> aType
foo Z Nil = something
foo (S pn) (h::t) = somethingElse
\end{code}


Also, computing a value or deciding an equality can potentially gives us information about multiple parameters of a function. For his purpose, Idris provides the "with" clause.

ADD EXAMPLE OF WITH CLAUSE HERE

\subsection{Provisional definitions}

blabla

\begin{code}[caption=Using decidable equality to infer the shape of parameters, captionpos=b, label=lst1:haskell2]
data Parity : Nat -> Set where
    even : Parity (plus n n)
    odd  : Parity (S (plus n n));

parity : (n:Nat) -> Parity n
parity Z = even {n=Z}
parity (S Z) = odd {n=Z}
parity (S (S k)) with parity k {
  parity (S (S (plus j j)))     
        | even ?= even {n=S j}  [paritySSe]
  parity (S (S (S (plus j j)))) 
        | odd  ?= odd {n=S j}   [paritySSo]
}
\end{code}

bla