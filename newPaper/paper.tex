%-----------------------------------------------------------------------------
%
%			Edwin's and Franck's Paper !
%  uses the template for sigplanconf LaTeX Class
%
%-----------------------------------------------------------------------------


\documentclass{sigplanconf}

% The following \documentclass options may be useful:

% preprint      Remove this option only once the paper is in final form.
% 10pt          To set in 10-point type instead of 9-point.
% 11pt          To set in 11-point type instead of 9-point.
% authoryear    To obtain author/year citation style instead of numeric.

\usepackage{amsmath}
\usepackage{amssymb}
\usepackage{graphicx}
\usepackage{listings}
\newcommand{\cL}{{\cal L}}


\begin{document}

\special{papersize=8.5in,11in}
\setlength{\pdfpageheight}{\paperheight}
\setlength{\pdfpagewidth}{\paperwidth}

\conferenceinfo{POPL ’15}{January 15-17, 2015, Mumbai, India.}
\copyrightyear{2014} 
\copyrightdata{978-1-nnnn-nnnn-n/yy/mm} 
\preprintfooter{Our paper for POPL '15}
\doi{nnnnnnn.nnnnnnn}

% Uncomment one of the following two, if you are not going for the 
% traditional copyright transfer agreement.

%\exclusivelicense                % ACM gets exclusive license to publish, 
                                  % you retain copyright

%\permissiontopublish             % ACM gets nonexclusive license to publish
                                  % (paid open-access papers, 
                                  % short abstracts)

\titlebanner{banner above paper title}        % These are ignored unless
\preprintfooter{short description of paper}   % 'preprint' option specified.

\title{Proof by reflection for equalities in algebraic structures}
%\subtitle{An example for a collection of tactics solving equalities in algebraic structures}

%\authorinfo{Edwin Brady\and Franck Slama}
%           {University of St Andrews, United-Kingdom}
%           {ecb10@st-andrews.ac.uk \and fs39@st-andrews.ac.uk}
\authorinfo{Author's name omitted for submission}
           {University of ..., Country}
           {email address omitted for submission}

\bibliographystyle{abbrvnat}

\maketitle

\nocite{*}



\begin{abstract}

We present a collection of reflective and correct by construction tactics for automatically proving equalities in different algebraic structures in the dependently typed
programming language Idris.
The distinctive feature of this correct by construction approach is that it makes a
strong link between the implementation of the tactic and its proof of correctness,
thereby enhancing the re-usability and the readability of the proof.
This example shows that the use of dependent types can be of great value for
building correct by construction software, since dependent types enable to capture
logical properties easily. Our approach is also generic in order to avoid code
duplication for the different algebraic structures : we are building a hierarchy of tactics where each tactic uses the others from the above levels.

%Before Edwin's help :

%We present a collection of reflective and correct by construction tactics for solving equalities in different algebraic structures for the Idris language. The main interest of this correct by construction approach is to make a strong link between the implementation of the tactic and its proof of correctness, thereby enhancing the re-usability and the readability of the proof, unlike the inflexible usual way which consists in programming and later proving the correctness. This example shows that a the use of dependent types can be of great value for building correct by construction software, since dependent types enable to capture easily logical properties. Our approach is also fairly generic in order to avoid code duplication for the different algebraic structures.

% My second abstract, still to verbose

%In the quest of certified software, at least two main approachs are possible. The usual way is generally to write a program, and then to prove its correctness afterwards. An alternative is to ensure the correction by construction, i.e., by having such precise types that all the properties we want are already ensured. Both approachs have their benefits. The usual way is naturally good for dealing with a lot of properties to ensure, one proved after the other, and all are proved after the definition of the function. The correct by construction approach is not so good for dealing with lot of properties to ensure, but one of its merits is that the writing of the program and the proof is simultaneously -forcing the programmer to think about correctness while programming-, and every bit of code has its "proof part", making therefore a strong link between the function and its proof of correctness. If the code of the function changes, that will be much easier to change the correspong bit in the proof. The usual approach usually fails for this purpose, since a script of proof done afterwards in a proof mode is not easy to maintain, mainly because these kind of scripts are not easy to read.
%In this paper, we propose to describe as an exemple a correct by construction implementation of a solver of equalities over some algebraic structures (magma, semigroup, monoid, group, ring and commutative ring). This kind of automatic solvers are useful for proof assistants because these proofs of equalities usually consist of a long and not really interesting serie of rewriting. Thanksfully, for each of the algebraics structures we mentionned, there exists a complete and decidable algorithm which determines if a given equality is true, and which constructs a proof of this fact if appropriate. Our approach is to work by reflection -enabling to write the tactic in the language itself-, and to ensure its correness by construction, thanks to our extensively use of dependent types. Also, our approach is fairly  generic and avoid a lot of duplication for the solvers over the different structures. 

%Previous abstract, too much focus on the solver, and not enought on the correct by construction part, wich is our contribution

%In the quest of certified sofware, one problem still remains: how can we make the process of proving easier? Some auxiliary proofs can be long and are not part of the interesting bit. For these cases, we would like to be able to let the machine find the proof for us. In this paper  we focus on a specific kind of  proofs : proofs of equalities over groups and rings. For this kind of structures, there exists a complete and decidable algorithm which determines if a given equality is true, and constructs a proof of this fact if so. However, since we want the tactic to be safe, we need to produce a proof of correction of this implementation. In our approach, this proof is obtained for free, because the tactic is built correct by construction, thanks to our extensive use of dependent types. Our approach is to work by reflection, and to normalize both side of the equality with a serie of rewritings that uses the axioms of the algebraic structures. At every step of the rewriting, a proof of equality between the new and the preivous terms is generated. Our approach is also generic, since it works for groups or ring, whether or not commutative. This example shows that the use of dependent types can help building software correct by construction.
\end{abstract}

\category{F.4.1}{Mathematical logic and formal languages}
			{Mathematical logic}
			[Proof theory]
\category{F.3.1}{Logics and meanings of programs}
			{Specifying and Verifying and Reasoning about Programs}
			[Mechanical verification]
\category{D.3.3}{Programming Languages}
			{Language Constructs and Features}
			[Abstract data types]


% general terms are not compulsory anymore, 
% you may leave them out
\terms
Languages, Verification

\keywords
dependent types, proof automation, reflection





Remark : To do : making the introduction more SPECIFIC, with CRUNCHY things : what we would like to do, no general assertions like "we often have to prove equalities...". (cf Simon P J talk)
\section{Introduction}


A very common situation in formal certification is to need to prove that one term is equal to another within some theories. Of course, they are usually not syntactically equal -this case would be trivial-, but they are equal according to a set of properties. These properties are either obtained by implementation (the implementation of + for relatives numbers is symmetric for example), or by axioms if we are working with an abstract structure.
These proofs of equality are usually done by hand, and are made with a potentially long sequence of rewriting, using the available properties that we have. For example, if $x$ and $y$ are relatives numbers, and if we want to show that :
$(x + y) * z = (z * x) + (y * z)$, we can first use the symmetry of $*$, saying that
$\forall a b, a * b = b * a$.
With the use of this property, we can rewrite the left part into the term $z * (x + y)$.
If now we use the distributivity of $*$ over $+$, which says that :
$\forall a b c, a * (b+c) = a*b + a*c$,
we now obtain the term $x*z + y*z$ for the left side.
If we use again the symmetry of $*$ on the subterm $(x*z)$, the left side of the equality becomes : $z*x + y*z$, which is what we want.





Thus, this kind of proofs consists of a potentially long sequence of rewriting, every rewriting step using one property of the theory. Without some specific automation, this sequence of rewriting is done by the user of the proof assistant. This is time consuming, and a little change in the left or the right hand side of the equality can invalidate completely the proof. The re-usability of this kind of proofs is indeed very low, since they are performing rewritings for a very specific term. In this paper, we describe a certified implementation of an automatic solver for equalities on algebraic structures, for the Idris language. Idris is a relatively new purely functional and dependently typed programming language with full dependent types. Idris also supports tactic based proofs. 

For our goal, we focus on some specific theories, which are the following algebraic structures : magmas, semi-group, monoid, group, Abelian group, ring and commutative ring.
For these structures, this is effectively possible to decide if a given equality is true, and to produce the proof of the equality is appropriate. This result comes directly from the fact that there exist a normal form for every element of these structures. The general idea is therefore to normalize both side of the equality, and then to compare the resulting terms using the usual syntactical Leibniz's equality.
This approach was followed by [paper Coq ring] for implementing a ring solver for the Coq proof assistant.

Our contribution mainly consists of two parts :
	- We follow a "correct by construction" approach for implementing the reduction procedures, instead of implementing a normalization procedure, and proving afterwards that this function effectively computes a normal form. For achieving this goal, we are extensively using dependent types in order to capture the interesting properties that matters for assuring the correctness of the method. 
	- We try to be as generic as possible, and re-use as much as possible the code of normalization from the structures below in the upper structures. For example developing an expression by using the axiom of distributivity works exactly the same in a commutative ring and in a ring, and this part of the reduction should be factorisable.


         
\section {A simpler problem : a little tactic correct by construction for lists (1/2 pages)}

We will start with the study of a smaller problem in which we aim to deal with variable lists, and associativity of list concatenation.
For example, we would like to be able to automatically generate proofs of facts like $\forall l1\ l2\ l3 : List,\ (l1 ++ l2) ++ (l3 ++ l4) = (l1 ++ (l2 ++ l3)) + l4$. \\
For this smaller problem, we have decided to only work with the property of associativity of the concatenation : $\forall l1\ l2\ l3,\ (l1 ++ l2) ++ l3 = l1 ++ (l2 ++ l3)$. \\
Thus, in definitive, in this section, we want to write a decision procedure, able to tell if two expressions composed of variable lists and concatenation of variable lists are equal, and to produce of proof of this equality if appropriate, when "equal" has the meaning "syntactically equal or equal thanks to associativity".

The general idea -which will also apply for the real bigger problem explained in the next sections- will be to normalize both sides of the "potential equality" $l1=l2$, and afterwards to compare them using Leibniz syntactical equality.
Of course, the normalization will have to fulfil the following property :
$\forall l, norm\ l = l$ [1], which means that after normalization we obtain a list which is provably equal to the input list (potentially using multiple time the property of associativity!). \\
After normalization, we will compare the two resulting lists, and if they are equal, we will have obtained the desired proof.  \\
Indeed :
$norm\ l1 = norm\ l2$ will imply $l1=l2$ by [1].

In fact, this is not exactly how it will work, because we will work by reflection, as we will do for the real problem later. The datatype presented in the Listing 1 is a first approximation of our reflected lists composed of variable lists, concatenation of lists and empty list.


\begin{lstlisting}[caption=Reflected lists, captionpos=b, label=lst1:haskell2]
data Expr : (Vect n (List a)) -> Type where
	App  : {n:Nat} -> {G : Vect n (List a)} -> Expr G -> Expr G -> Expr G 
	Var  : {n:Nat} -> (G : Vect n (List a)) -> (i : Fin n) -> Expr G
   ENil : {n:Nat} -> (G : Vect n (List a)) -> Expr G 
\end{lstlisting}

For this smaller problem, we want to prove equalities in which lists are always universally quantified, so we only need to represent variable (a universally quantified list), the constant list Nil, and the concatenation between lists.
Variables are represent using a "De Brujin" index : (Var fZ) denotes a variable, (Var (fS fZ)) another one, and so on.

The type Expr is indexed over a vector of lists (we will usually call it G), that represents our context. For example, if we want to prove $\forall l1\ l2\ l3 : List,\ (l1 ++ l2) ++ (l3 ++ l4) = (l1 ++ (l2 ++ l3)) + l4$, then we will have to encode $(l1 ++ l2) ++ (l3 ++ l4)$ and $(l1 ++ (l2 ++ l3)) + l4$ in a context of four elements. The first element of this context denotes the variable $l1$, the second denotes $l2$, and so on.
Thus, the left hand side will be encoded by :
\begin{lstlisting}[caption=Reflected LHS, captionpos=b, label=lst1:haskell2]
App (App (Var G fZ) 
			(Var G (fS fZ))) 
    (App (Var G (fS (fS fZ))) 
    	   (Var G (fS (fS (fS fZ))))).
\end{lstlisting}

We need now to define what will be the normalization procedure.
If it is a function which takes an Expr and produces another Expr, then we will need to prove the following lemma afterwards :
$\forall\ le1,\ interpretation\ (reduce le1)\ =\ interpretation\ le1$
where $interpretation$ is a function computing the interpretation of an Expr, that is to say, the list this Expr is encoding.

This proof can be quite tricky to make because it relies on the complete behaviour of the reduction and on the way we compute the interpretation.
For this little example, the reduction will not be too heavy, but in the next sections, when we will have more properties than only associativity to deal with, it will certainly become more problematic to "unfold" the definition of a gigantic reduction procedure. \\
To avoid these two sources of complexity, we will :
- Use dependent types in order to directly capture the concrete list that an Expr is encoding
- Write a "correct by construction" reduction procedure. The proof of preservation of the interpretation will be computed bit by bit at the same time as the function will produce the normalized expression.

\begin{lstlisting}[caption=New reflected lists, captionpos=b, label=lst1:haskell2]
using (x : List a, y : List a, G : Vect n (List a))
  data Expr : (G : Vect n (List a)) -> List a -> Type where
       App  : Expr G x -> Expr G y -> Expr G (x ++ y)
       Var  : (i : Fin n) -> Expr G (index i G)
       ENil : Expr G []
\end{lstlisting}

The reduction procedure is now supposed to be written on a "correct by construction" way, which mean that no additional proof should required after the definition of the function. Thus, $reduce$ will produce the proof that the new Expr produced has the same interpretation as the original Expr, and this will be made easier by the fact that the datatype Expr is now indewed over the real - concrete list : a term of type $Expr G l$ is the encoding of the list $l$.
Thus, we can write the type of $reduce$ like this : \\
$reduce\ :\ Expr\ G\ x\ ->\ (x'\ **\ (Expr\ G\ x',\ x\ =\ x'))$ \\
The function $reduce$ produces a dependent pair : the new concrete list $x'$, and a pair composed by an $Expr G x'$ which is the new encoded term indexed iver the new concrete list, and a proof that old and new lists are equal.
 










\section {Back to the general problem : A hierachy of tactics (2/3 pages)}

	\subsection {Hierarchy of typeclass}

	\subsection {Reflected terms}

	\subsection {Deciding equality}

\section {The details : normalization functions (4/5 pages)}

	\subsection {Normalization of terms in a Magma}

	\subsection {Normalization of terms in a Semi-Group}

	\subsection {Normalization of terms in a Monoid}

	\subsection {Normalization of terms in a Group}

\section {Auxiliary problems : Automatic metaification and coating (1/2 pages)}

	\subsection {Automatic metaification}

	\subsection {Coating : transforming this stuff into real tactics}

\section {Related work (1 page)}

\section {Conlusion (0,5/1 page)}


\acks

Acknowledgments : Chris, Mattus...

% We recommend abbrvnat bibliography style.


\bibliography{biblio}

\appendix
\section{Appendix Title}

This is the text of the appendix, if you need one.

\end{document}

%                       Revision History
%                       -------- -------
%  Date         Person  Ver.    Change
%  ----         ------  ----    ------

%  2013.06.29   TU      0.1--4  comments on permission/copyright notices

