%-----------------------------------------------------------------------------
%
%			Edwin's and Franck's Paper !
%  uses the template for sigplanconf LaTeX Class
%
%-----------------------------------------------------------------------------


\documentclass{sigplanconf}

% The following \documentclass options may be useful:

% preprint      Remove this option only once the paper is in final form.
% 10pt          To set in 10-point type instead of 9-point.
% 11pt          To set in 11-point type instead of 9-point.
% authoryear    To obtain author/year citation style instead of numeric.

\usepackage{amsmath}
\usepackage{amssymb}
\usepackage{graphicx}
\usepackage{listings}
\usepackage[literate]{idrislang} 
\newcommand{\cL}{{\cal L}}

%\lstset {captionpos=b}

\begin{document}

\special{papersize=8.5in,11in}
\setlength{\pdfpageheight}{\paperheight}
\setlength{\pdfpagewidth}{\paperwidth}

\conferenceinfo{POPL ’15}{January 15-17, 2015, Mumbai, India.}
\copyrightyear{2014} 
\copyrightdata{978-1-nnnn-nnnn-n/yy/mm} 
\preprintfooter{Our paper for POPL '15}
\doi{nnnnnnn.nnnnnnn}




% Uncomment one of the following two, if you are not going for the 
% traditional copyright transfer agreement.

%\exclusivelicense                % ACM gets exclusive license to publish, 
                                  % you retain copyright

%\permissiontopublish             % ACM gets nonexclusive license to publish
                                  % (paid open-access papers, 
                                  % short abstracts)

\titlebanner{banner above paper title}        % These are ignored unless
\preprintfooter{short description of paper}   % 'preprint' option specified.

\title{Proof by reflection for equalities in algebraic structures}
%\subtitle{An example for a collection of tactics solving equalities in algebraic structures}

%\authorinfo{Edwin Brady\and Franck Slama}
%           {University of St Andrews, United-Kingdom}
%           {ecb10@st-andrews.ac.uk \and fs39@st-andrews.ac.uk}
\authorinfo{Author's name omitted for submission}
           {University of ..., Country}
           {email address omitted for submission}

\bibliographystyle{abbrvnat}

\maketitle

\nocite{*}



\begin{abstract}

We present a collection of reflective and correct by construction tactics for automatically proving equalities in different algebraic structures in the dependently typed
programming language Idris.
The distinctive feature of this correct by construction approach is that it makes a
strong link between the implementation of the tactic and its proof of correctness,
thereby enhancing the re-usability and the readability of the proof.
This example shows that the use of dependent types can be of great value for
building correct by construction software, since dependent types enable to capture
logical properties easily. Our approach is also generic in order to avoid code
duplication for the different algebraic structures : we are building a hierarchy of tactics where each tactic uses the others from the above levels.

%Before Edwin's help :

%We present a collection of reflective and correct by construction tactics for solving equalities in different algebraic structures for the Idris language. The main interest of this correct by construction approach is to make a strong link between the implementation of the tactic and its proof of correctness, thereby enhancing the re-usability and the readability of the proof, unlike the inflexible usual way which consists in programming and later proving the correctness. This example shows that a the use of dependent types can be of great value for building correct by construction software, since dependent types enable to capture easily logical properties. Our approach is also fairly generic in order to avoid code duplication for the different algebraic structures.

% My second abstract, still to verbose

%In the quest of certified software, at least two main approachs are possible. The usual way is generally to write a program, and then to prove its correctness afterwards. An alternative is to ensure the correction by construction, i.e., by having such precise types that all the properties we want are already ensured. Both approachs have their benefits. The usual way is naturally good for dealing with a lot of properties to ensure, one proved after the other, and all are proved after the definition of the function. The correct by construction approach is not so good for dealing with lot of properties to ensure, but one of its merits is that the writing of the program and the proof is simultaneously -forcing the programmer to think about correctness while programming-, and every bit of code has its "proof part", making therefore a strong link between the function and its proof of correctness. If the code of the function changes, that will be much easier to change the correspong bit in the proof. The usual approach usually fails for this purpose, since a script of proof done afterwards in a proof mode is not easy to maintain, mainly because these kind of scripts are not easy to read.
%In this paper, we propose to describe as an exemple a correct by construction implementation of a solver of equalities over some algebraic structures (magma, semigroup, monoid, group, ring and commutative ring). This kind of automatic solvers are useful for proof assistants because these proofs of equalities usually consist of a long and not really interesting serie of rewriting. Thanksfully, for each of the algebraics structures we mentionned, there exists a complete and decidable algorithm which determines if a given equality is true, and which constructs a proof of this fact if appropriate. Our approach is to work by reflection -enabling to write the tactic in the language itself-, and to ensure its correness by construction, thanks to our extensively use of dependent types. Also, our approach is fairly  generic and avoid a lot of duplication for the solvers over the different structures. 

%Previous abstract, too much focus on the solver, and not enought on the correct by construction part, wich is our contribution

%In the quest of certified sofware, one problem still remains: how can we make the process of proving easier? Some auxiliary proofs can be long and are not part of the interesting bit. For these cases, we would like to be able to let the machine find the proof for us. In this paper  we focus on a specific kind of  proofs : proofs of equalities over groups and rings. For this kind of structures, there exists a complete and decidable algorithm which determines if a given equality is true, and constructs a proof of this fact if so. However, since we want the tactic to be safe, we need to produce a proof of correction of this implementation. In our approach, this proof is obtained for free, because the tactic is built correct by construction, thanks to our extensive use of dependent types. Our approach is to work by reflection, and to normalize both side of the equality with a serie of rewritings that uses the axioms of the algebraic structures. At every step of the rewriting, a proof of equality between the new and the preivous terms is generated. Our approach is also generic, since it works for groups or ring, whether or not commutative. This example shows that the use of dependent types can help building software correct by construction.
\end{abstract}

\category{F.4.1}{Mathematical logic and formal languages}
			{Mathematical logic}
			[Proof theory]
\category{F.3.1}{Logics and meanings of programs}
			{Specifying and Verifying and Reasoning about Programs}
			[Mechanical verification]
\category{D.3.3}{Programming Languages}
			{Language Constructs and Features}
			[Abstract data types]


% general terms are not compulsory anymore, 
% you may leave them out
\terms
Languages, Verification

\keywords
dependent types, proof automation, reflection





Remark : To do : making the introduction more SPECIFIC, with CRUNCHY things : what we would like to do, no general assertions like "we often have to prove equalities...". (cf Simon P J talk)
\section{Introduction}


A very common situation in formal certification is to need to prove that one term is equal to another within some theories. Of course, they are usually not syntactically equal -this case would be trivial-, but they are equal according to a set of properties. These properties are either obtained by implementation (the implementation of + for relatives numbers is symmetric for example), or by axioms if we are working with an abstract structure.
These proofs of equality are usually done by hand, and are made with a potentially long sequence of rewriting, using the available properties that we have. For example, if $x$ and $y$ are relatives numbers, and if we want to show that :
$(x + y) * z = (z * x) + (y * z)$, we can first use the symmetry of $*$, saying that
$\forall a b, a * b = b * a$.
With the use of this property, we can rewrite the left part into the term $z * (x + y)$.
If now we use the distributivity of $*$ over $+$, which says that :
$\forall a b c, a * (b+c) = a*b + a*c$,
we now obtain the term $x*z + y*z$ for the left side.
If we use again the symmetry of $*$ on the subterm $(x*z)$, the left side of the equality becomes : $z*x + y*z$, which is what we want.





Thus, this kind of proofs consists of a potentially long sequence of rewriting, every rewriting step using one property of the theory. Without some specific automation, this sequence of rewriting is done by the user of the proof assistant. This is time consuming, and a little change in the left or the right hand side of the equality can invalidate completely the proof. The re-usability of this kind of proofs is indeed very low, since they are performing rewritings for a very specific term. In this paper, we describe a certified implementation of an automatic solver for equalities on algebraic structures, for the Idris language. Idris is a relatively new purely functional and dependently typed programming language with full dependent types. Idris also supports tactic based proofs. 

For our goal, we focus on some specific theories, which are the following algebraic structures : magmas, semi-group, monoid, group, Abelian group, ring and commutative ring.
For these structures, this is effectively possible to decide if a given equality is true, and to produce the proof of the equality is appropriate. This result comes directly from the fact that there exist a normal form for every element of these structures. The general idea is therefore to normalize both side of the equality, and then to compare the resulting terms using the usual syntactical Leibniz's equality.
This approach was followed by [paper Coq ring] for implementing a ring solver for the Coq proof assistant.

Our contribution mainly consists of two parts :
	- We follow a "correct by construction" approach for implementing the reduction procedures, instead of implementing a normalization procedure, and proving afterwards that this function effectively computes a normal form. For achieving this goal, we are extensively using dependent types in order to capture the interesting properties that matters for assuring the correctness of the method. 
	- We try to be as generic as possible, and re-use as much as possible the code of normalization from the structures below in the upper structures. For example developing an expression by using the axiom of distributivity works exactly the same in a commutative ring and in a ring, and this part of the reduction should be factorisable.


         
\section {A smaller problem : a correct by construction tactic for the associativity of lists concatenation}

In order to better explain the way we are trying to solve our problem, we will first present it on a simplified version, in which we aim to deal with universaly quantified lists, and associativity of list concatenation.
For example, we would like to be able to automatically generate proofs of facts like $\forall l1\ l2\ l3 : List,\ (l1 ++ l2) ++ (l3 ++ l4) = (l1 ++ (l2 ++ l3)) + l4$ [Example 1]. \\
For this smaller problem, we have decided to only work with the associativity of the concatenation : $\forall l1\ l2\ l3,\ (l1 ++ l2) ++ l3 = l1 ++ (l2 ++ l3)$ and with the fact that Nil is a neutral element for the concatenation : $\forall l, l ++ Nil = l$. Of course, this work could be extended by dealing with more axioms, and this is exactly what we will do in the next sections when we will generalize this problem. \\
Thus, in definitive, in this section, we want to write a decision procedure, able to tell if two expressions composed of universally quantified lists and concatenation of these lists are equal, and to produce of proof of this equality if appropriate, when "equal" has the meaning "syntactically equal or equal thanks to associativity or thanks to the "neutrality" of Nil". \\

When trying to prove this kind of equalities, the variables are abstracted, and they become part of the context. For the example 1, after abstraction of the variables, the goal becomes simply $(l1 ++ l2) ++ (l3 ++ l4) = (l1 ++ (l2 ++ l3)) + l4$, which is something of the general form $x=y$.
The general idea -which will also apply for the real bigger problem detailed in the next sections- will be to normalize both sides of the "potential equality" $x=y$, and afterwards to compare them using Leibniz syntactical equality.
The goal of the normalization is of course to compute a canonical representation for a list $l$, such that any other list provably equal to $l$ (by using the associativity) will have the same canonical representation. That's why it will be possible to decide the equality by simply comparing the two normal forms with a week syntactical equality. \\
\\
Of course, the normalization will have to fulfil the following correctness property :
$\forall l, norm\ l = l$ [1], which means that after normalization we obtain a list which is provably equal to the input list (potentially using the property of associativity multiple times). \\
And after normalization, we will compare the two resulting lists with a standard syntactical equality, and if they are equal, we will have obtained the desired proof.  \\
Indeed :
$norm\ x = norm\ y$ will imply $x=y$ by using the correctness lemma [1] of $norm$ two times. \\
\\
In fact, this is not exactly how it will work. This way would not be easily feasible, because in the LHS and RHS of $l1=l2$, we potentially have variables which have been universally quantified. And the normalization function would need to do different treatments for a "variable list" (a list which has been universally quantified) and for a constant empty list. This is not possible yet, because once the variables are abstracted, nothing tells us where the variables are in the left and right hand side of the equality : this is a meta information of the AST that we can't access.
For this reason, we will work by reflection, as we will do for for the real problem later. That means that we will define a datatype which will be used as an encoding of lists, or more precisely, an encoding of lists composed of "variable lists", Nil, and concatenation of the previous. This datatype will allow us to know the internal structure of the list, ie, where the variables and constants are.
Indeed, this datatype will allow pattern matching. Previously, we were only able to pattern match a list against the constructor $Nil$ and $Cons$, which wasn't interesting for our goal.  With the first approximation of the datatype $Expr$ presented in the Listing 1, we will be allowed to pattern match an encoding of list against the constructor App, Var and ENil, which gives us the informations we want.

\begin{code}[caption=First version of reflected lists, captionpos=b, label=lst1:haskell2]
data Expr : (Vect n (List a)) -> Type where
  App  : {n:Nat} -> {G : Vect n (List a)} -> 
         Expr G -> Expr G -> Expr G 
  Var  : {n:Nat} -> (G : Vect n (List a)) -> 
         (i : Fin n) -> Expr G
  ENil : {n:Nat} -> (G : Vect n (List a)) -> 
         Expr G 
\end{code}

For this smaller problem, we want to prove equalities in which lists are always universally quantified, so we only need to represent variable (a universally quantified list), the constant list Nil, and the concatenation between lists.
Variables are represent using a De Brujin-like index : (Var fZ) denotes a variable, (Var (fS fZ)) another one, and so on.

The type Expr is indexed over a vector of lists (we will usually name it G), that represents our context. For example, if in the example 1, we will have to encode $(l1 ++ l2) ++ (l3 ++ l4)$ and $(l1 ++ (l2 ++ l3)) + l4$ in a context where four elements are declared. The first element of this context denotes the variable $l1$, the second denotes $l2$, and so on.
Thus, the left hand side will be encoded by :
\begin{code}[caption=Reflected LHS of example 1, captionpos=b, label=lst1:haskell2]
App (App (Var G fZ) 
         (Var G (fS fZ))) 
    (App (Var G (fS (fS fZ))) 
         (Var G (fS (fS (fS fZ)))))
\end{code}

We need now to define what will be the normalization procedure.
If, as it is probably expected, this function simply takes an Expr and produces another Expr, then we will need to prove the following lemma afterwards, corresponding to the correcness property mentioned before : \\
$\forall\ e:Expr\ G,\ interpretation\ (reduce\ e)\ =\ interpretation\ e$ \\
where $interpretation$ is a function computing the interpretation of an Expr, that is to say, the list this Expr is encoding.
This proof can be quite tricky to make because it relies on the complete behaviour of the reduction and on the way we compute the interpretation.
For this little example, the reduction will not be too heavy, but in the next sections, when we will have more properties than only associativity to deal with, it will certainly become more problematic to "unfold" the definition of a gigantic reduction procedure. \\
\\
To avoid these two sources of complexity, we will : \\
- Use dependent types in order to directly capture and embed the concrete list that an Expr is encoding. Thus, it won't be necessary to define the $interpretation$ function. \\
- Write a "correct by construction" reduction procedure. The proof of preservation of the interpretation will be computed "bit by bit" at the same time as the function will produce the normalized expression (by rewriting the term thanks to the property of associativity).

The type for reflected lists therefore becomes :
\begin{code}[caption=Second version of reflected lists with embedded denotation, captionpos=b, label=lst1:haskell2]
using (x : List a, y : List a, 
       G : Vect n (List a))
  data Expr : (G : Vect n (List a)) -> 
              List a -> Type where
       App  : Expr G x -> Expr G y -> 
              Expr G (x ++ y)
       Var  : (i : Fin n) -> 
              Expr G (index i G)
       ENil : Expr G []
\end{code}
For an expression $e\ :\ Expr\ G\ x$, we will say that "$e$ denotes (or encodes) the list $x$".
When an expression is a variable, the denoted list is simply the corresponding variable in the context, ie, $(index\ i\ G)$.
Also, the $ENil$ expression denotes the $Nil$ list.
Finally, if $e1$ is an expression encoding the list $x$, and $e2$ is an expression encoding the list $y$, then the expression $App\ e1\ e2$ denotes the list $(x ++ y)$.

The reduction procedure is now supposed to be written on a "correct by construction" way, which means that no additional proof should be required after the definition of the function. Thus, $reduce$ will produce the proof that the new Expr produced has the same interpretation as the original Expr, and this will be made easier by the fact that the datatype Expr is now indexed over the real (concrete) list : a term of type $Expr\ G\ l$ is the encoding of the list $l$.
Thus, we can write the type of $reduce$ like this : \\
$reduce\ :\ Expr\ G\ x\ \rightarrow\ (x'\ **\ (Expr\ G\ x',\ x\ =\ x'))$ \\
The function $reduce$ produces a dependent pair : the new concrete list $x'$, and a pair made of an $Expr\ G\ x'$ which is the new encoded term indexed by the new concrete list, and a proof that old and new -real- lists are equal.
Note that this function can't simply produce an $Expr\ G\ x$, because the list on which the resulting expression will be indexed is not necessary syntactically equal to the original list since it can use the property of associativity. 
In fact, what really interest us in this function is precisely the proof of $x\ =\ x'$.
The reason is simple : when we try to automatically prove $x=y$, these proofs $x=x'$ and $y=y'$ will be the crucial part for the construction of the desired proof. \\
\\
We will have an expression $e1$ encoding $x$, and an expression $e2$ encoding $y$\footnote{These encodings have to be produced by hand by the user for this small simple problem, but for the real collection of tactics, we will program an automatic metaification}.
We will normalize $e1$, and this will give a new list $x'$, a new expression $e1':Expr\ G\ x'$, and a proof of $x=x'$. We will do the same with $e2$, and we will get a new list $y$, an expression $e2':Expr\ G\ y'$, and a proof of $y=y'$. \\
Now, we can compare $e1'$ and $e2'$ using a standard syntactical equality because they are in their normal form :

\begin{code}[caption=Syntactical equality, captionpos=b, label=lst1:haskell2]
  eqExpr : (e : Expr G x) -> (e' : Expr G y) 
  	       -> Maybe (e = e')
  eqExpr (App x y) (App x' y') 
          with (eqExpr x x', eqExpr y y')
    eqExpr (App x y) (App x y)   
    	| (Just refl, Just refl) = Just refl
    eqExpr (App x y) (App x' y') 
    	| _ = Nothing
  eqExpr (Var i) (Var j) with (decEq i j)
    eqExpr (Var i) (Var i) 
    	| (Yes refl) = Just refl
    eqExpr (Var i) (Var j) 
    	| _ = Nothing
  eqExpr ENil ENil = Just refl
  eqExpr _ _ = Nothing
\end{code}


And of course, if $e1'$ and $e2'$ are equal, then they necessary have the same type\footnote{We are working with the heterogeneous equality JMeq, but as always, the only way to have a proof of a:A = b:B is when A=B}, and therefore $x'=y'$
By rewriting the two previously obtained equality $x=x'$ and $y=y'$ in the new equality $x'=y'$, we can get a proof of $x=y$.

\begin{code}[caption=Building the desired proof with the two proofs of equality, captionpos=b, label=lst1:haskell2]
  buildProof : {x : List a} -> {y : List a} 
               -> Expr G ln -> Expr G rn ->
               (x = ln) -> (y = rn) ->
               Maybe (x = y) 
  buildProof e e' lp rp with (eqExpr e e')
    buildProof e e lp rp  | Just refl = ?bp1
    buildProof e e' lp rp | Nothing = Nothing
\end{code}

As we said, the proof for the metavariable is just the rewriting of the two equalities :

\begin{code}[caption=buildProof metavariable, captionpos=b, label=lst1:haskell2]
  bp1 = proof {
  intros;
  refine Just;
  rewrite sym lp;
  rewrite sym rp;
  exact refl;
}  
\end{code}
Finally, the main function which tries to prove the equality $x=y$ is simply made of the reduction of the two metaified terms reflecting the left and the right hand side, and of the composition of the two obtained proofs thanks to the function $buildProof$:
\begin{code}[caption=testEq, captionpos=b, label=lst1:haskell2]
  testEq : Expr G x -> Expr G y 
           -> Maybe (x = y)
  testEq l r = 
     let (ln ** (l', lPrf)) = reduce l in 
     let (rn ** (r', rPrf)) = reduce r in
        buildProof l' r' lPrf rPrf
\end{code}
Now, we need to define the function reduce. To do that, we have to fix a canonical representation of associative lists. We decide that the left associative form will be the canonical representation. Thus, the $reduce$ function will have to rewrite the metaified term by rearranging the parentheses in order to transform the underlying list in the form $(...((l1 ++ l2) ++ l3) ... ++ ln)$. To do so, one possibility is to define a new datatype which captures this property, and to write a function going from $Expr$ to this new type. Thus it will be easier to be certain that we are effectively computing the normal form : forcing properties to hold by the shape of a datatype is a good usage of dependent types when, like here, it doesn't introduce more complications.
\begin{code}[caption=Reflected left associative lists, captionpos=b, label=lst1:haskell2]
  data LExpr : (G : Vect n (List a)) -> 
               List a -> Type where
       LApp : LExpr G x -> (i : Fin n) -> 
              LExpr G (x ++ index i G)
       LNil : LExpr G []
\end{code}
This datatype has only two constructors. In fact, it combines the previous $Var$ and $App$ constructors so that it becomes impossible to write an expression which isn't left associative.
 
As part of the normalization, we will write a function $expr\_l$ converting an $Expr\ G\ x$ to a $LExpr\ G\ x'$ and producing a proof that $x=x'$. This function will therefore use the property of associativity multiple times, in order to obtain the expected fully left associative form. Of course, because by using this new datatype $LExpr$ we've changed the representation of our encoded lists, we will need to convert back an $LExpr\ G\ x$ to an $Expr\ G\ x$. The function $l\_expr$ will do this easy task.

\begin{code}[caption=Production of the left associative form, captionpos=b, label=lst1:haskell2]
  expr_l : Expr G x -> 
               (x' ** (LExpr G x', x = x'))
  expr_l ENil = (_ ** (LNil, refl))
  expr_l (Var i) = (_ ** 
                   (LApp RNil i, refl))
  expr_l (App ex ey) = 
  	let (xl ** (xr, xprf)) = expr_l ex in
  	let (yl ** (yr, yprf)) = expr_l ey in
   	   appLExpr _ _ xr yr xprf yprf
    where 
       appLExpr : (x', y' : List a) ->
         LExpr G x -> LExpr G y -> 
         (x' = x) -> (y' = y) ->
         (w' ** (LExpr G w', x' ++ y' = w'))
       appLExpr x' y' rxs (LApp e i) xprf yprf =
          let (xs ** (rec, prf)) = 
             appLExpr _ _ rxs e refl refl in
             (_ ** (LApp rec i, ?appLExpr1))
       appLExpr x' y' rxs LNil xprf yprf = 
          (_ ** (rxs, ?appLExpr2))


  l_expr : LExpr G x -> Expr G x
  l_expr LNil = ENil
  l_expr (LApp xs i) = App (r_expr xs) (Var i)
\end{code}

We notice that for transforming the list into its left associative equivalent representation, we've effectively needed to know where the variables and the Nil constants are : the functions $expr\_l$ and $l\_expr$ are doing different treatments for these different possibilities. \\
\\
We've got two metavariables to prove. The metavariable $appRExpr1$ requires us to prove the goal : $x' ++ y' = xs ++ index\ i\ G$ in a context where we've got, amongst other things,  $(xprf\ :\ x'\ =\ x)$, $(yprf\ :\ y'\ =\ x2\ ++\ index\ i\ G)$ and $(prf\ :\ x\ ++\ x2\ =\ xs)$.
Proving this goal uses the property of associativity after rewriting the goal with these three proof of equality $xprf$, $yprf$ and $prf$.

\begin{code}[caption=Proof of the metavariable appRExpr1, captionpos=b, label=lst1:haskell2]
AutoAssoc.appRExpr1 = proof {
  intros;
  rewrite sym xprf;
  rewrite sym yprf;
  rewrite prf;
  rewrite sym 
       (appendAssociative x x2 (index i G));
  exact refl;
}
\end{code}

And the metavariable appRExpr2 uses the fact that Nil is a neutral element for the concatenation.

\begin{code}[caption=Proof of the metavariable appRExpr2, captionpos=b, label=lst1:haskell2]
appRExpr2 = proof {
  intros;
  rewrite xprf;
  rewrite sym yprf;
  rewrite appendNilRightNeutral x';
  exact refl;
}
\end{code}

We can now define the reduction, which is just the composition of the two previous functions $expr\_l$ and $l\_expr$:

\begin{code}[caption=Reduction function, captionpos=b, label=lst1:haskell2]
  reduce : Expr G x -> 
           (x' ** (Expr G x', x = x'))
  reduce e = 
     let (x' ** (e', prf)) = expr_l e in
        (x' ** (l_expr e', prf))
\end{code}

At the moment, what we've got is not exactly a real tactic, in the sense that we only have a function which produces a value of type $Maybe (x = y)$. A real tactic would be a coating of this function, which could properly fail when the two terms are not equal. However, here, when $x\ne y$, the function $testEq$ will simply produce the value $Nothing$. \\
It's now time to see how to use this minimalist "tactic".
Let's define two expressions $e1$ and $e2$, respectively representing the lists $((xs ++ ys) ++ (xs ++ zs))$ and $(xs ++ ((ys ++ xs) ++ zs))$ in the context $[xs, ys, zs]$ of three abstracted variables.

\begin{code}[caption=Two test expressions, captionpos=b, label=lst1:haskell2]
  e1 : (xs, ys, zs : List a) -> 
           Expr [xs, ys, zs] 
           ((xs ++ ys) ++ (xs ++ zs))
  e1 xs ys zs = 
     App (App (Var fZ) (Var (fS fZ))) 
         (App (Var fZ) (Var (fS (fS fZ))))

  e2 : (xs, ys, zs : List a) -> 
           Expr [xs, ys, zs] 
           (xs ++ ((ys ++ xs) ++ zs))
  e2 xs ys zs = 
     App (Var fZ) 
         (App (App (Var (fS fZ)) (Var fZ)) 
              (Var (fS (fS fZ))))
\end{code}

The lists denoted by the expressions $e1$ and $e2$ are equal, and we can generate a proof of this by using $testEq$.
\begin{code}[caption=Test of equality betwen e1 and e2, captionpos=b, label=lst1:haskell2]
  e1_e2_testEq : (xs, ys, zs : List a) ->
          Maybe (((xs ++ ys) ++ (xs ++ zs)) = 
                 (xs ++ ((ys ++ xs) ++ zs)))
  e1_e2_testEq xs ys zs = 
     testEq (e1 xs ys zs) (e2 xs ys zs) 
\end{code}

And if we ask for the evaluation of this term, we should obtain $Just$ and a proof of equality between the two underlying lists.
\begin{code}[caption=Evaluation of the result, captionpos=b, label=lst1:haskell2]
\xs => \ys => \zs => e1_e2_testEq {a=Int} xs ys zs

\xs => \ys => \zs => Just (replace (sym (replace 
(sym (replace (replace (appendNilRightNeutral xs)
 refl (replace (sym (appendAssociative xs [] ys)) 
 refl))) (replace (sym (replace (replace 
 (appendNilRightNeutral [...] refl)))) refl)) 
: (xs : List Int) -> (ys : List Int) -> 
  (zs : List Int) -> 
  Maybe ((xs ++ ys) ++ xs ++ zs 
         = xs ++ (ys ++ xs) + zs)
\end{code}

And we effectively get what we wanted : a proof of equality between the two lists. As expected, this proof uses the properties of associativity ($appendAssociative$) and the property of neutrality of $Nil$ for $append$ ($appendNilRightAssociative$).

\section {Back to the general problem : A hierachy of tactics (2/3 pages)}

	\subsection {Hierarchy of typeclass}

	\subsection {Reflected terms}

	\subsection {Deciding equality}

\section {The details : normalization functions (4/5 pages)}

	\subsection {Normalization of terms in a Magma}

	\subsection {Normalization of terms in a Semi-Group}

	\subsection {Normalization of terms in a Monoid}

	\subsection {Normalization of terms in a Group}

\section {Auxiliary problems : Automatic metaification and coating (1/2 pages)}

	\subsection {Automatic metaification}

	\subsection {Coating : transforming this stuff into real tactics}

\section {Related work (1 page)}

\section {Conlusion (0,5/1 page)}


\acks

Acknowledgments : Chris, Mattus for discussions. Jan for help in latex.

% We recommend abbrvnat bibliography style.


\bibliography{biblio}

\appendix
\section{Appendix Title}

This is the text of the appendix, if you need one.

\end{document}

%                       Revision History
%                       -------- -------
%  Date         Person  Ver.    Change
%  ----         ------  ----    ------

%  2013.06.29   TU      0.1--4  comments on permission/copyright notices

