\section{The General Problem: A Hierarchy of Tactics}

\label{sect:general}

Now that we have introduced our ideas on this little example, it is time to apply them for our main goal, more general : the implementation of a hierarchy of tactics proving equalities in algebraic structures. Very often, the properties available on a given type are the ones of a well known algebraic structure, like semi-group, monoid, group, ring...  We will no longer only work with the properties of associativity and of right neutral element for natural numbers as it was the case in the previous section, but we will have available the properties of different structures, and these tactics will be usable for any type that satisfies these properties.

We will construct a hierarchy of type classes and will write one tactic for each of these type class. All our tactics (the ring prover, the group prover...) will be able to work on any type, being given that an instance of the corresponding type class is provided.

\subsection {Hierarchy of type classes}

In fact, we don't only want to be able to prove equalities, but more generally equivalence. We want to give the user the possibility to define his own equivalence notion, assuming he's able to provide the properties of the structure he wants to use. This equivalence relation on $c$ has the following profile $(\simeq) : c \rightarrow c \rightarrow Type$\footnote{This 'Type' would be a 'Prop' in systems --like Coq-- that make a distinction between the world of computations and the world of logical statements}, and has to be accompanied by the usual properties of reflexivity, symmetry and transitivity.

All of our tactics will require to have a way of testing this equivalence between elements of the underlying set, that is to say, a way to test equivalence between constants. For this reason, we define a notion of ``set\footnote{This notion of ``Set" isn't a formalisation of sets and their elements, but is only the Setoid that we need} ", which only requires the definition of the equivalence relation --accompanied with the proofs that this is indeed an equivalence relation-- and this equivalence test $set\_eq$. All the algebraic structures will later extend this typeclass :

\begin{figure}[H]
\figrule
\begin{center}
\begin{lstlisting}
class Set c where
    ($\simeq$) : c $\rightarrow$ c $\rightarrow$ Type

    refl : (x:c) $\rightarrow$ x $\simeq$ x
    sym : {x:c} $\rightarrow$ {y:c} $\rightarrow$ (x $\simeq$ y) $\rightarrow$ (y $\simeq$ x)
    trans : {x:c} $\rightarrow$ {y:c} $\rightarrow$ {z:c} $\rightarrow$ (x $\simeq$ y) $\rightarrow$ (y $\simeq$ z) $\rightarrow$ (x $\simeq$ z)    
    
    set_eq : (x:c) $\rightarrow$ (y:c) $\rightarrow$ Maybe (x $\simeq$ y)
\end{lstlisting}
\end{center}
\caption{Set}
\label{Set}
\figrule
\end{figure}
If the user wants to prove propositional equalities, then he will simply instantiate $(\simeq)$ with the built-in $(=)$ during the definition of the instance of $Set$.
Note that $(\simeq)$ is only weakly decidable in the sense that $set\_eq$ only produces a proof when the two elements are equivalent, but it doesn't produce a proof of dis-equivalence when they are different -- instead, it simply produces the value $Nothing$--. That's quite natural, since we want to generate proof of equivalence, and not to generate counter examples for proving dis-equivalence, which is another problem.

Obviously, there is no tactic associated to $Set$, since we have no operations and no properties associated to this structure. Therefore, equivalence in a $Set$ are just the "syntactical equivalence", and they can simply be proven with $refl$.
The first real structure, almost trivial, is the magma. A magma has just a $Plus$ operation, and no properties about it.

\begin{figure}[H]
\figrule
\begin{center}
\begin{lstlisting}
class Set c => Magma c where
    Plus : c $\rightarrow$ c $\rightarrow$ c
\end{lstlisting}
\end{center}
\caption{Magma}
\label{Magma}
\figrule
\end{figure}

This code means that a type $c$ (for $carrier$) is a Magma if it is already a Set (ie, it is equiped with the equivalence relation $\simeq$ and the test $set\_eq$), and if it has a $Plus$ operation.
In fact, there is an additional requirement which is that all operations, like the Plus here, need to be "compatible" with the equality : $Plus\_preserves\_equiv : \{c1:c\} \rightarrow \{c2:c\} \rightarrow \{c1':c\} \rightarrow \{c2':c\} \rightarrow (c1 \simeq c1') \rightarrow (c2 \simeq c2') \rightarrow ((Plus\ c1\ c2) \simeq (Plus\ c1'\ c2'))$. This requirement comes from the fact that we're dealing with any equivalence relation. The user is free to use his own equivalence relation, but it should be compatible with the operations that he is using.

A bit more interesting is the Semi-Group typeclass. A Semi-Group is a Magma (ie, it still has a $Plus$ operation), but moreover it has the property of associativity for this operation.

\begin{figure}[H]
\figrule
\begin{center}
\begin{lstlisting}
class Magma c => SemiGroup c where
    Plus_assoc : (c1:c) $\rightarrow$ (c2:c) $\rightarrow$ (c3:c) 
               -> (Plus (Plus c1 c2) c3 $\simeq$ Plus c1 (Plus c2 c3))
\end{lstlisting}
\end{center}
\caption{Semi-Group}
\label{SemiGroup}
\figrule
\end{figure}

And the Monoid structure is a Semi-Group with the property of neutral element.

\begin{figure}[H]
\figrule
\begin{center}
\begin{lstlisting}
class SemiGroup c => Monoid c where
    Zero : c    
    Plus_neutral_1 : (c1:c) $\rightarrow$ (Plus Zero c1 $\simeq$ c1)    
    Plus_neutral_2 : (c1:c) $\rightarrow$ (Plus c1 Zero $\simeq$ c1)
\end{lstlisting}
\end{center}
\caption{Monoid}
\label{Monoid}
\figrule
\end{figure}
We can continue the hierarchy with the more interesting structure of Group.
A group is a monoid, with two new operations. The binary operation $Minus$, and the unary operation $Neg$. We must have the property that $Minus$ can always be simplified with the $Neg$ and the $Plus$. That means that $Minus$ is not a "primitive" operation of a Group, since we can always rewrite $(a-b)$ into $(a\ +\ (-b))$. We let the possibility to express a $Minus$ just for convenience for the user, so that he won't have to rewrite by hand his $Minus$ operations with sums of negations.
The second axiom, which is the most important one for a Group, is the fact that any value $c1$ admits $Neg\ c1$ for inverse, where being an additive inverse is the following property :

\begin{figure}[H]
\figrule
\begin{center}
\begin{lstlisting}
-- This is a conjunctive predicate
hasSymmetric : (c:Type) $\rightarrow$ (p:Monoid c) $\rightarrow$ c $\rightarrow$ c $\rightarrow$ Type
hasSymmetric c p a b $\simeq$ (Plus a b $\simeq$ Zero, Plus b a $\simeq$ Zero)    
  
class Monoid c => Group c where
    Minus : c $\rightarrow$ c $\rightarrow$ c
    Neg : c $\rightarrow$ c
    Minus_simpl : (c1:c) $\rightarrow$ (c2:c) $\rightarrow$ Minus c1 c2 $\simeq$ Plus c1 (Neg c2) 
    Plus_inverse : (c1:c) $\rightarrow$ hasSymmetric c _ c1 (Neg c1)
\end{lstlisting}
\end{center}
\caption{Group}
\label{Group}
\figrule
\end{figure}


This hierarchy can be extended without difficulty with Abelian groups (with the axiom of commutativity), Ring, Commutative Rings...

These typeclasses will somehow be used as predicates which classify types. In order to call a prover on a type $c$, the user of the system will have to satisfy the corresponding typeclass by providing an instance of it for his type $c$, ie, he will have to prove that the corresponding properties effectively hold for $c$.

Note that the properties will either be obtained by implementation if the operations are real --computable-- functions, or by axioms if the user is working in an axiomatised theory where the operations ($Plus$, $Neg$, ...) are defined as axioms.

As discussed in the section 2.1 on the smaller problem, the algorithm of normalization will not directly use the concrete value of type $c$, but a reflected term, indexed over the concrete value. That still holds here.


	\subsection {Reflected terms}

We need to define a datatype for reflecting terms in each algebraic structure.
Each of these datatype is parametrised over a type $c$, which is the real type on which we want to prove equalities (the $carrier$ type). It is also indexed over an instance of the corresponding typeclass for $c$ (we usually call it $p$, because it behaves as a $proof$ telling that the structure $c$ has the desired properties), and indexed over a context (a vector $\Gamma$ of $n$ elements of type $c$), and also indexed over a value of type $c$, which is precisely the concrete value being encoded.
A magma is only equipped with one operation $Plus$. Thus, we only have three concepts to express in order to reflect terms in a Magma : constants, variables, and additions.



\begin{figure}[H]
\figrule
\begin{center}
\begin{lstlisting}
data ExprMa : Magma c $\rightarrow$ (Vect n c) $\rightarrow$ c $\rightarrow$ Type where
    ConstMa : (p : Magma c) $\rightarrow$ ($\Gamma$:Vect n c) $\rightarrow$ (c1:c)  $\rightarrow$ ExprMa p $\Gamma$ c1 
    PlusMa : {p : Magma c} $\rightarrow$ {$\Gamma$:Vect n c} $\rightarrow$ {c1:c} $\rightarrow$ {c2:c} 
         $\rightarrow$ ExprMa p $\Gamma$ c1 $\rightarrow$ ExprMa p $\Gamma$ c2 
         $\rightarrow$ ExprMa p $\Gamma$ (Plus c1 c2) 
    VarMa : (p:Magma c) $\rightarrow$ {$\Gamma$:Vect n c}
         $\rightarrow$ (i:Fin n) $\rightarrow$ ExprMa p $\Gamma$ (index i $\Gamma$)
\end{lstlisting}
\end{center}
\caption{Reflected terms for a Magma}
\label{ExprMa}
\figrule
\end{figure}

When we encode a constant $c1$ in a context $\Gamma$, we use the constructor $ConstMa$ to produce a term of type $ExprMa\ p\ \Gamma\ c1$ : the index representing the concrete value is precisely this constant $c1$.
If e1 is an expression of type $ExprMa\ p\ \Gamma\ c1$ (ie, a term encoding the value $c1$), and $e2$ is an expression of type $ExprMa\ p\ \Gamma\ c2$ (ie, a term encoding the value $c2$), then the term $PlusMa\ e1\ e2$ will have the type $ExprMa\ p\ \Gamma\ (Plus\ c1\ c2)$, ie, this term will encode the value $(Plus\ c1\ c2)$, where $Plus$ is the operation available in the current instance $p$.


Because there is no additional operations in SemiGroup and Monoid, the datatypes for reflected terms in these two structures will have exactly the same shape as the one for Magma that we've given above.
However, the one for $Group$ will introduce two new constructors for the $Neg$ and $Minus$ operations :


\begin{figure}[H]
\figrule
\begin{center}
\begin{lstlisting}
data ExprG :  Group c $\rightarrow$ (Vect n c) $\rightarrow$ c $\rightarrow$ Type where
    ConstG : (p : Group c) $\rightarrow$ ($\Gamma$:Vect n c) 
         -> (c1:c) $\rightarrow$ ExprG p $\Gamma$ c1
    PlusG : {p : Group c} $\rightarrow$ {$\Gamma$:Vect n c} $\rightarrow$ {c1:c} $\rightarrow$ {c2:c} 
         $\rightarrow$ ExprG p $\Gamma$ c1 $\rightarrow$ ExprG p $\Gamma$ c2 
         $\rightarrow$ ExprG p $\Gamma$ (Plus c1 c2)
    MinusG : {p : Group c} $\rightarrow$ {$\Gamma$:Vect n c} $\rightarrow$ {c1:c} $\rightarrow$ {c2:c} 
         $\rightarrow$ ExprG p $\Gamma$ c1 $\rightarrow$ ExprG p $\Gamma$ c2 
         $\rightarrow$ ExprG p $\Gamma$ (Minus c1 c2)
    NegG : {p : Group c} $\rightarrow$ {$\Gamma$:Vect n c} $\rightarrow$ {c1:c} 
         $\rightarrow$ ExprG p $\Gamma$ c1 $\rightarrow$ ExprG p $\Gamma$ (Neg c1)
    VarG : (p : Group c) $\rightarrow$ {$\Gamma$:Vect n c} 
         $\rightarrow$ (i:Fin n) $\rightarrow$ ExprG p $\Gamma$ c1
\end{lstlisting}
\end{center}
\caption{Reflected terms in a Group}
\label{ExprG}
\figrule
\end{figure}

The index of type $c$ (the real value represented by an expression) is always expressed by using the available operations in the type class $p$, which for a group Group are $Plus$, $Minus$ and $Neg$.

	\subsection {A bit of notation}
The equality we are trying to prove is $x \simeq y$, where $x$ and $y$ are elements of the type $c$, which  simulates a set with some properties (which make $c$ being a Semi-Group, or a Monoid, or a Group...). The fact that $c$ fulfils the specification of an algebraic structure is expressed as an instance of the corresponding type class, and this instance will be denoted as $p$.
The reflected term for $x$ will be denoted $ex$, and this term will have the type $ExprG\ p\ \Gamma\ x$. The term $ex$ is the encoding of $x$ and its type is precisely indexed over the real value $x$, as the reflected terms embed the concrete values.
We've got similar thing for $y$, which is encoded by $ey$, and its type is indexed over the real value $y$.
Running the normalisation procedure on $ex$ will produce the normal form $ex'$ of type $ExprG\ p\ \Gamma\ x'$ and a proof $px$ of $x \simeq x'$ while running the normalisation procedure on $ey$ will produce the normal form $ey'$ of type $ExprG\ p\ \Gamma\ y'$ and a proof $py$ of $y \simeq y'$.

	\subsection {Deciding equivalence}
	
Each algebraic structure will have a function for reducing the reflected terms into their normal forms. The fact that all of these algebraic structures admit a canonical represent for any element is in fact a very nice property that we are using in order to decide equalities. Without this property, it would become a lot more complicated to decide the equality without brute-forcing a serie of rewriting, that might even not terminate.
These normalisation functions will compute the canonical representation of the reflected input term, by rewriting the given term multiple times, and at the same time, it will produce the proof of equality between the original real value (indexing the input term) and the produced real value (indexing the produced term). Thus, the type of the reduction function for group is :


\begin{lstlisting}
groupReduce : {c:Type} $\rightarrow$ {n:Nat} $\rightarrow$ (p:Group c) $\rightarrow$ {$\Gamma$:Vect n c} 
	   $\rightarrow$ {x:c} $\rightarrow$ (ExprG p $\Gamma$ x) $\rightarrow$ (x' ** (ExprG p $\Gamma$ x', x $\simeq$ x'))
\end{lstlisting}


This function has more work to do in structures with multiple axioms (like Group), than for the easier structure underneath.
The details of these functions will be given in the next section. For the moment, we just assume that we've got such a function for each structure. We continue to give here the details for the Group structure, but the principle is exactly the same with the other structures, and the exact equivalent of what we did in the previous section with natural numbers and additions.

We want to write the following function :


\begin{lstlisting}
groupDecideEq : (p:Group c) $\rightarrow$ {$\Gamma$:Vect n c} $\rightarrow$ {x : c} $\rightarrow$ {y : c} 
	     $\rightarrow$ (ExprG p $\Gamma$ x) $\rightarrow$ (ExprG p $\Gamma$ y) 
	     $\rightarrow$ Maybe (x $\simeq$ y)
\end{lstlisting}

The first thing this function has to do is to compute the normal form of the two expressions $ex$ and $ey$ respectively encoding $x$ and $y$.
Then, the only thing remaining to do will be to compare syntactically $ex'$ and $ey'$, and if they are equal, then we will get $x'=y'$, and we will be able to produce the desired proof of $x \simeq y$ by using the two proofs $px : x \simeq x'$ and $py : y \simeq y'$. 


\begin{lstlisting}
groupDecideEq p ex ey =
  let (x' ** (ex', px)) = groupReduce p ex in
  let (y' ** (ey', py)) = groupReduce p ey in
	     buildProofGroup p ex' ey' px py
\end{lstlisting}


The syntactical test of equality between $ex'$ and $ey'$ and the composition of the two proofs is done in the auxiliary function $buildProofGroup$, similarly to what we've done on the smaller example in the previous section. 


\begin{lstlisting}
buildProofGroup : (p:Group c) $\rightarrow$ {$\Gamma$:Vect n c} 
  $\rightarrow$ {x : c} $\rightarrow$ {y : c} $\rightarrow$ {x':c} $\rightarrow$ {y':c} 
  $\rightarrow$ (ExprG p $\Gamma$ x') $\rightarrow$ (ExprG p $\Gamma$ y') 
  $\rightarrow$ (x $\simeq$ x') $\rightarrow$ (y $\simeq$ y') 
  $\rightarrow$ (Maybe (x $\simeq$ y))
buildProofGroup p ex' ey' px py with (exprG_eq p ex' ey')
    buildProofGroup p ex' ex' px py | Just refl = ?MbuildProofGroup
    buildProofGroup p ex' ey' px py | Nothing = Nothing
\end{lstlisting}

We've used here the function $exprG\_eq$ to decide the syntactical equality between the normalised reflected terms $ex'$ and $ey'$. This function plays the same role as $eqExpr$, presented on section 2.3 for the smaller example. Also, the proof for the metavariable $MbuildProofGroup$ corresponds exactly to what was done for the metariable $MbuildProof$ in the smaller example.


\subsection {Automatic reflection}

As we did for the very specific problem in section 2, we want to program an automatic reflection mechanism, in order to let the machine build automatically the encodings for us. This time, we want to write a function able to compute encodings for any type that behaves as one of our algebraic structure. We will show here the details for Ring, which is one of the most sophisticated structures. So, we've got a carrier type $(c:Type)$, an instance $(p:Ring\ c)$ which says that $c$ behaves as a ring, a context $\Gamma$ of already abstracted variables, and an element $(x:c)$ to encode, and we want to return an $ExprR\ p\ \Gamma\ x$.
We are doing this reflection for any type $c$, so we don't know which constants are inhabiting this type. For this reason, we will need to expect an extra argument --$funReflectCsf$-- of type $(ConstantsRingReflector\ c\ p)$ which is a function able to do the reflection for constants only. We will be able to provide this function for various datatypes, like $Nat$, $Z$ and $List$. But if the user wants to use the reflection mechanism for a new datatype of this own, then he will only have to provide this function of encoding for the constants of his type. He won't have to rebuild the entire reflection mechanism.
A $ConstantsRingReflector$ for a given type $c$ and a given instance $p$ is a function that takes any context $\Gamma$, any element of $c$, and which tries to produce an $(ExprR\ p\ \Gamma\ x)$.


\begin{lstlisting}
ConstantsRingReflector : {c:Type} $\rightarrow$ (p:Ring c) $\rightarrow$ Type
ConstantsRingReflector {c=c} p = 
     ({n:Nat} $\rightarrow$ ($\Gamma$:Vect n c) $\rightarrow$ (x:c) $\rightarrow$ Maybe (ExprR p $\Gamma$ x))
\end{lstlisting}


If $x$ is indeed a constant, then it is supposed to return $Just$ and a term reflecting this constant. If $x$ was something else (a variable, a sum $a+b$, a product $a*b$, ...) then it simply returns $Nothing$. \\
\\
We can now present the type of the reflection function for ring.


\begin{lstlisting}
reflectRingTerm : {c:Type} $\rightarrow$ {p:Ring c} $\rightarrow$ {n:Nat} $\rightarrow$ ($\Gamma$ : Vect n c) 
     $\rightarrow$ (funReflectCsf:ConstantsRingReflector {c=c} p) $\rightarrow$ (x:c) 
     $\rightarrow$ (n' ** ($\Gamma'$:Vect n' c ** (ExprR {c=c} {n=n+n'} p ($\Gamma$ ++ $\Gamma'$) x)))
\end{lstlisting}


This function will work very similarly to the reflection function for $Nat$ that we've writen in section~\ref{sect:ReflectNat}. Here, there are simply more constructions to deal with, because $c$ is also equipped with negations, subtractions and multiplications, and not only with additions. We show one of these new pattern on figure~{\ref{reflectRingTerm_pattern1}.

\begin{figure}[H]
\figrule
\begin{center}
\begin{lstlisting}
reflectRingTerm {p} {n=n} $\Gamma$ funReflectCsf (a*b) = 
   let (n' ** ($\Gamma'$ ** a')) = reflectRingTerm $\Gamma$ funReflectCsf a in
   let (n'' ** ($\Gamma''$ ** b')) = reflectRingTerm ($\Gamma$ ++ $\Gamma'$) funReflectCsf b in
   let this = MultR (weakenR $\Gamma''$ a') b' in
	((n' + n'') ** (($\Gamma'$++$\Gamma''$) 
	  ** (convertVectInExprR (plusAssociative n n' n'')
	                   (vectAppendAssociative $\Gamma$ $\Gamma'$ $\Gamma''$) this)))
\end{lstlisting}
\end{center}
\caption{Automatic reflection - case of a multiplication}
\label{reflectRingTerm_pattern1}
\figrule
\end{figure}

After having done the same thing with additions, negations and subtractions, we will just have the case of constants and variable missing. But since we've got $funReflectCsf$ --the function that tries to do the reflection for constants only--, we can easily finish the definition of $reflectRingTerm$. If the dedicated reflector $funReflectCsf$ has returned something, then we know the input is a variable, and we have its encoding. Otherwise, we just have to treat it as a variable.

\begin{figure}[H]
\figrule
\begin{center}
\begin{lstlisting}
reflectRingTerm {p} {n=n} $\Gamma$ funReflectCsf t =
    case funReflectCsf $\Gamma$ t of
      -- If 'funReflectCsf' has decided that 't' is a constant, then we trust it
      Just this => Z ** ([] ** this)
      -- If not, then it should be considered as a variable
      Nothing => case (isElement t $\Gamma$) of
          Just (i ** pr) => let res = VarR p (RealVariable i) in
			        (Z ** ([ ] ** res))
          Nothing => let res = VarR p (RealVariable (lastElement' n)) in
			 (S Z ** ([t] ** res))

\end{lstlisting}
\end{center}
\caption{Automatic reflection - case of constants and variables}
\label{reflectRingTerm_pattern2}
\figrule
\end{figure}

We've omitted to show a few conversions of type in order to keep this part of the definition more readable. \\
\\
It is remarkable that the reflected term is guaranteed to be a faithful representation of the input. We've been able to define this function with a very strong and precise type because Idris has a ``$\%syntax$" command which let us do arbitrary pattern matching on syntax, and the functions defined with ``$\%syntax$" are typed, unlike Ltac terms in Coq.





