\section {Results}

We've implemented a hierarchy of tactics for proving equalities in various algebraic structures. The implementation uses a type-safe reflection mechanism where reflected expressions are indexed over idris expressions, which enables to pull out the proof easily from the index and to write the reduction procedures on a correct-by-construction way. This approach removes much of the difficulty of the construction of the proof. All the proofs we had to do during our implementation were straightforward.

In section 1, we've presented a small example with binary numbers and their addition, for which the user had to prove the lemma $adc\_lemma\_2$ by hand. We're now almost ready to use our commutative monoid prover for building automatically a proof of the desired equality.

\begin{figure}[H]
\figrule
\begin{center}
\begin{lstlisting}
plus (plus c (plus bit (plus v (plus v 0)))) (plus bit (plus v (plus v 0))) 
   = plus v2 (plus (plus (plus v1 v) v) (plus (plus (plus v1 v) v) 0))
\end{lstlisting}
\end{center}
\figrule
\end{figure} 

In a context we're we've got the following induction hypothesis \[ihn:(plus\ (plus\ c\ bit)\ bit1\ =\ plus\ vLsb\ (plus\ vCarry0\ (plus\ vCarry0\ 0)))\]
The goal can't be solved directly by calling commutative monoid prover, as it uses different variables in its left and right hand side. However, the induction hypothesis $ihn$ --obtained with the recursive call-- creates a link between these variables. Therefore, we can use our reduction procedure for commutative monoid four times : two for reducing the LHS and RHS of $ihn$, and two for reducing the LHS and RHS of the goal. After these four normalisations, it becomes possible to simply rewrite the new induction impothesis $ihn'$ on the current goal.

\begin{tikzpicture}
  \matrix (m) [matrix of math nodes,row sep=3em,column sep=4em,minimum width=2em]
  {
     LHS_{ihn} & RHS_{ihn} \\
     LHS'_{ihn} & RHS'_{ihn} \\};
  \path
    (m-1-1) edge [-stealth] node [left] {$norm$}
    						node [right] {$=$} (m-2-1)
            
    (m-1-1.east|-m-1-2) edge node [above] {$=$} 
    						 node [below] {$ihn$} (m-1-2)
    						 
    (m-2-1.east|-m-2-2) edge [dashed,-] node [above] {$=$}
										node [below] {$ihn'$} (m-2-2) 
    
    (m-1-2) edge [-stealth] node [right] {$norm$}
    						node [left] {$=$} (m-2-2);  
\end{tikzpicture}
\begin{tikzpicture}
  \matrix (m) [matrix of math nodes,row sep=3em,column sep=4em,minimum width=2em]
  {
     LHS_{goal} & RHS_{goal} \\
     LHS'_{goal} & RHS'_{goal} \\
     LHS''_{goal} \\};
  \path
    (m-1-1) edge [-stealth] node [left] {$norm$}
							node [right] {$=$} (m-2-1)    
    
            edge [dashed,-] node [above] {$?=$} (m-1-2)

    (m-2-1.east|-m-2-2) edge [dashed,-] node [below] {$?=$} (m-2-2)
    (m-2-1) edge [-stealth] node [left] {$ihn'$}
							node [right] {$=$} (m-3-1)  
    
    (m-1-2) edge [-stealth] node [left] {$=$}  
							node [right] {$norm$} (m-2-2)
    
    (m-3-1) edge [double] node [right] {$=$} (m-2-2);    
\end{tikzpicture}

This example was a bit unconventional in the sense that 4 calls to the normalisation function were needed. Usually, only two calls to the normalisation are enough, one for the left hand side and one for the right hand side, and they are done in the decision procedure $commutativeMonoidDecideEq$. These two extra calls were needed because of the known equality that we had from the induction hypothesis. This example\footnote{Which can be found in $RingIdris/Examples/binary.idr$}, which was therefore at the boundary to deduction modulo [REF], shows that our implementation can deal with concrete goals obtained during the development of real programs.
