% paper.tex

\NeedsTeXFormat{LaTeX2e}

\documentclass{jfp1}

\usepackage{float}

%%% Macros for the guide only %%%
\providecommand\AMSLaTeX{AMS\,\LaTeX}
\newcommand\eg{\emph{e.g.}\ }
\newcommand\etc{\emph{etc.}}
\newcommand\bcmdtab{\noindent\bgroup\tabcolsep=0pt%
  \begin{tabular}{@{}p{10pc}@{}p{20pc}@{}}}
\newcommand\ecmdtab{\end{tabular}\egroup}
\newcommand\rch[1]{$\longrightarrow\rlap{$#1$}$\hspace{1em}}
\newcommand\lra{\ensuremath{\quad\longrightarrow\quad}}

\title[Automatically proving equivalence by type-safe reflection]
      {Automatically proving equivalence by type-safe reflection}

 \author[Franck Slama and Edwin Brady]
        {FRANCK SLAMA and EDWIN BRADY\\
         University of St Andrews, United Kingdom\\
         \email{fs39@st-andrews.ac.uk, ecb10@st-andrews.ac.uk}}

\jdate{December 2015}
\pubyear{2016}
\pagerange{\pageref{firstpage}--\pageref{lastpage}}
\doi{S0956796801004857}

\newtheorem{lemma}{Lemma}[section]

\begin{document}

\label{firstpage}

\maketitle

\begin{abstract}
% The abstract should be at most 300 words for the Journal of Functional Programming
Idris is a general purpose purely functional programming language with dependent types, aiming to bring type-based program verification techniques to functional programmers. One common difficulty with programming with dependent types is that proof obligations arise naturally once programs become even moderately sized. For example, implementing an adder for binary numbers indexed over their natural number equivalents will naturally lead to proof obligations for equalities of expressions over natural numbers. 
As far as possible, we would like to solve such proof obligations automatically. In this paper, we show one way to automate such proofs by reflection. We will show how representing Idris expressions in a reflected form, indexed by the original Idris expression, leads to straightforward construction and manipulation of proofs.
With this type-safe reflection technique, the resulting expressions are guaranteed to be faithful representations of the corresponding inputs and any generated proof is guaranteed to be a proof of the required property. 
We will first present the technique on a small example, where we aim to automatically prove equalities with universally quantified lists and the associative concatenation operation, and later show how this idea is generalised to various kind of properties (or axioms) that might be available (for example : commutativity, distributivity, existence of neutral elements...), leading to a hierarchy of reflexive tactics for Monoids, Commutative Monoids, Groups, Commutative Groups, Rings, and so on, written in Idris, for proving different kind of equivalence. We will also show how each tactic reuses the other ones from the simplest structures, thus avoiding as much as possible the duplication of code.
\end{abstract}

%\tableofcontents

\section{Introduction}


A very common situation in formal certification is to need to prove that one term is equal to another within some theories. Of course, they are usually not syntactically equal -this case would be trivial-, but they are equal according to a set of properties. These properties are either obtained by implementation (the implementation of + for relatives numbers is symmetric for example), or by axioms if we are working with an abstract structure.
These proofs of equality are usually done by hand, and are made with a potentially long sequence of rewriting, using the available properties that we have. For example, if $x$ and $y$ are relatives numbers, and if we want to show that :
$(x + y) * z = (z * x) + (y * z)$, we can first use the symmetry of $*$, saying that
$\forall a b, a * b = b * a$.
With the use of this property, we can rewrite the left part into the term $z * (x + y)$.
If now we use the distributivity of $*$ over $+$, which says that :
$\forall a b c, a * (b+c) = a*b + a*c$,
we now obtain the term $x*z + y*z$ for the left side.
If we use again the symmetry of $*$ on the subterm $(x*z)$, the left side of the equality becomes : $z*x + y*z$, which is what we want.





Thus, this kind of proofs consists of a potentially long sequence of rewriting, every rewriting step using one property of the theory. Without some specific automation, this sequence of rewriting is done by the user of the proof assistant. This is time consuming, and a little change in the left or the right hand side of the equality can invalidate completely the proof. The re-usability of this kind of proofs is indeed very low, since they are performing rewritings for a very specific term. In this paper, we describe a certified implementation of an automatic solver for equalities on algebraic structures, for the Idris language. Idris is a relatively new purely functional and dependently typed programming language with full dependent types. Idris also supports tactic based proofs. 

For our goal, we focus on some specific theories, which are the following algebraic structures : magmas, semi-group, monoid, group, Abelian group, ring and commutative ring.
For these structures, this is effectively possible to decide if a given equality is true, and to produce the proof of the equality is appropriate. This result comes directly from the fact that there exist a normal form for every element of these structures. The general idea is therefore to normalize both side of the equality, and then to compare the resulting terms using the usual syntactical Leibniz's equality.
This approach was followed by [paper Coq ring] for implementing a ring solver for the Coq proof assistant.

Our contribution mainly consists of two parts :
	- We follow a "correct by construction" approach for implementing the reduction procedures, instead of implementing a normalization procedure, and proving afterwards that this function effectively computes a normal form. For achieving this goal, we are extensively using dependent types in order to capture the interesting properties that matters for assuring the correctness of the method. 
	- We try to be as generic as possible, and re-use as much as possible the code of normalization from the structures below in the upper structures. For example developing an expression by using the axiom of distributivity works exactly the same in a commutative ring and in a ring, and this part of the reduction should be factorisable.



\begin{thebibliography}{}
 \bibitem[\protect\citename{Augustsson and Johnsson, }1987]{AJ187}
   Augustsson,~L. and Johnsson,~T. (1987) LML users' manual. PMG
   Report, Department of Computer Science, Chalmers University of
   Technology, Goteborg, Sweden.
 \bibitem[\protect\citename{Butcher, }1981]{Butcher}
   Butcher,~J. (1981) Copy-editing: the Cambridge handbook.
   Cambridge University Press.
 \bibitem[\protect\citename{Chicago, }1982]{Chicago}
   The Chicago manual of style. University of Chicago Press.
 \bibitem[\protect\citename{Conklin, }1987]{JC87}
   Conklin,~J. (1987) Hypertext: an introduction and survey.
   \emph{IEEE Computer}, 20~(9): pp.~17--41.
 \bibitem[\protect\citename{Dijkstra, }1976]{EWD76}
   Dijkstra,~E.~W. (1976) \emph{A Discipline of Programming}.
   Prentice-Hall.
 \bibitem[\protect\citename{Knuth, }1984]{DEK84}
   Knuth,~D.~E. (1984) Literate programming. \emph{BCS Comput. J.}
   27~(2): 97--111 (May).
 \bibitem[\protect\citename{Lamport, }1986]{LaTeX}
   Lamport,~L. (1986) \LaTeX: a document preparation system
   (2nd edition). Addison-Wesley, New York.
 \bibitem[\protect\citename{Reynolds, }1969]{JCR69}
   Reynolds,~J.~C. (1969) Transformation systems and the
   algebraic structure of atomic formulas. In B. Meltzer and
   D. Michie (editors), \emph{Machine Intelligence 5}, pp.~135--151.
   Edinburgh University Press.
 \bibitem[\protect\citename{Toyn \emph{et al.}, }1987]{TDR87}
   Toyn,~I., Dix,~A. and Runciman,~C. (1987) Performance
   polymorphism. In \emph{Functional Programming Languages and
   Computer Architecture, Lecture Notes in Computer Science, 274},
   pp.~325--346. Springer-Verlag.
\end{thebibliography}

\label{lastpage}

\end{document}

% end of JFP2egui.tex
