% paper.tex

\NeedsTeXFormat{LaTeX2e}

\documentclass{jfp1}

\usepackage{float}

%%% Macros for the guide only %%%
\providecommand\AMSLaTeX{AMS\,\LaTeX}
\newcommand\eg{\emph{e.g.}\ }
\newcommand\etc{\emph{etc.}}
\newcommand\bcmdtab{\noindent\bgroup\tabcolsep=0pt%
  \begin{tabular}{@{}p{10pc}@{}p{20pc}@{}}}
\newcommand\ecmdtab{\end{tabular}\egroup}
\newcommand\rch[1]{$\longrightarrow\rlap{$#1$}$\hspace{1em}}
\newcommand\lra{\ensuremath{\quad\longrightarrow\quad}}

\title[Automatically proving equivalence by type-safe reflection]
      {Automatically proving equivalence by type-safe reflection}

 \author[Franck Slama and Edwin Brady]
        {FRANCK SLAMA and EDWIN BRADY\\
         University of St Andrews, United Kingdom\\
         \email{fs39@st-andrews.ac.uk, ecb10@st-andrews.ac.uk}}

\jdate{December 2015}
\pubyear{2016}
\pagerange{\pageref{firstpage}--\pageref{lastpage}}
\doi{S0956796801004857}

\newtheorem{lemma}{Lemma}[section]

\begin{document}

\label{firstpage}

\maketitle

\begin{abstract}
% The abstract should be at most 300 words for the Journal of Functional Programming
Idris is a general purpose purely functional programming language with dependent types, aiming to bring type-based program verification techniques to functional programmers. One common difficulty with programming with dependent types is that proof obligations arise naturally once programs become even moderately sized. For example, implementing an adder for binary numbers indexed over their natural number equivalents will naturally lead to proof obligations for equalities of expressions over natural numbers. 
As far as possible, we would like to solve such proof obligations automatically. In this paper, we show one way to automate such proofs by reflection. We will show how representing Idris expressions in a reflected form, indexed by the original Idris expression, leads to straightforward construction and manipulation of proofs.
With this type-safe reflection technique, the resulting expressions are guaranteed to be faithful representations of the corresponding inputs and any generated proof is guaranteed to be a proof of the required property. 
We will first present the technique on a small example, where we aim to automatically prove equalities with universally quantified lists and the associative concatenation operation, and later show how this idea is generalised to various kind of properties (or axioms) that might be available (for example : commutativity, distributivity, existence of neutral elements...), leading to a hierarchy of reflexive tactics for Monoids, Commutative Monoids, Groups, Commutative Groups, Rings, and so on, written in Idris, for proving different kind of equivalence. We will also show how each tactic reuses the other ones from the simplest structures, thus avoiding as much as possible the duplication of code.
\end{abstract}

%\tableofcontents

\section{Introduction}


A very common situation in formal certification is to need to prove that one term is equal to another within some theories. Of course, they are usually not syntactically equal -this case would be trivial-, but they are equal according to a set of properties. These properties are either obtained by implementation (the implementation of + for relatives numbers is symmetric for example), or by axioms if we are working with an abstract structure.
These proofs of equality are usually done by hand, and are made with a potentially long sequence of rewriting, using the available properties that we have. For example, if $x$ and $y$ are relatives numbers, and if we want to show that :
$(x + y) * z = (z * x) + (y * z)$, we can first use the symmetry of $*$, saying that
$\forall a b, a * b = b * a$.
With the use of this property, we can rewrite the left part into the term $z * (x + y)$.
If now we use the distributivity of $*$ over $+$, which says that :
$\forall a b c, a * (b+c) = a*b + a*c$,
we now obtain the term $x*z + y*z$ for the left side.
If we use again the symmetry of $*$ on the subterm $(x*z)$, the left side of the equality becomes : $z*x + y*z$, which is what we want.





Thus, this kind of proofs consists of a potentially long sequence of rewriting, every rewriting step using one property of the theory. Without some specific automation, this sequence of rewriting is done by the user of the proof assistant. This is time consuming, and a little change in the left or the right hand side of the equality can invalidate completely the proof. The re-usability of this kind of proofs is indeed very low, since they are performing rewritings for a very specific term. In this paper, we describe a certified implementation of an automatic solver for equalities on algebraic structures, for the Idris language. Idris is a relatively new purely functional and dependently typed programming language with full dependent types. Idris also supports tactic based proofs. 

For our goal, we focus on some specific theories, which are the following algebraic structures : magmas, semi-group, monoid, group, Abelian group, ring and commutative ring.
For these structures, this is effectively possible to decide if a given equality is true, and to produce the proof of the equality is appropriate. This result comes directly from the fact that there exist a normal form for every element of these structures. The general idea is therefore to normalize both side of the equality, and then to compare the resulting terms using the usual syntactical Leibniz's equality.
This approach was followed by [paper Coq ring] for implementing a ring solver for the Coq proof assistant.

Our contribution mainly consists of two parts :
	- We follow a "correct by construction" approach for implementing the reduction procedures, instead of implementing a normalization procedure, and proving afterwards that this function effectively computes a normal form. For achieving this goal, we are extensively using dependent types in order to capture the interesting properties that matters for assuring the correctness of the method. 
	- We try to be as generic as possible, and re-use as much as possible the code of normalization from the structures below in the upper structures. For example developing an expression by using the axiom of distributivity works exactly the same in a commutative ring and in a ring, and this part of the reduction should be factorisable.


\section{A simplified problem : type safe reflection for proving equalities with universally quantified natural numbers and additions}


In order to better explain the way we are trying to solve our problem, we will first present it on a simplified version, in which we aim to deal with universally quantified natural numbers, the properties of associativity and neutral element for the operation plus.
For example, we would like to be able to automatically generate proofs of goals like $\forall x1\ x2\ x3:Nat,\ (x1 + l2) + (x3 + x4) = (x1 + (x2 + x3)) + x4$ [Example 1]. \\
For this smaller problem, we have decided to only work with the associativity of plus : $\forall x1\ x2\ x3,\ (x1 + x2) + x3 = x1 + (x2 + x3)$ and with the fact that Z is a neutral element on the right side of the plus : $\forall x, x + Z = x$. Zero is also a left neutral element, but this property is not needed, as we have this behavirous by reduction as $+$ is defined recursively on its first argument. Of course, some structures can have more or less properties than these two, and this work will be extended in the next sections when we will write a general hierarchy of solvers for different algebraic structures. \\
Thus, in definitive, in this section, we want to write a decision procedure, able to tell if two expressions composed of universally quantified natural numbers and additions of these numbers are equal, and to produce of proof of this equality if appropriate, when "equal" has the meaning "syntactically equal or equal thanks to the associative and neutral properties". \\


\subsection{Working by reflexion}

When trying to prove this kind of equalities, the variables are abstracted, and they become part of the context. On the example 1, after abstraction of the variables, the goal becomes simply $(x1 + x2) + (x3 + x4) = (x1 + (x2 + x3)) + x4$, which is something of the general form $x=y$.
The general idea --which will also apply for the more general and bigger problem detailed in the next sections-- will be to normalize both sides of the "potential equality" $x=y$, and afterwards to compare them using Leibniz syntactical equality.
The goal of the normalization is to compute a canonical representation for a number $x$, such that any other number provably equal to $x$ (by using the two available properties) will have the same canonical representation. For example, the normalisation might transform $x+((y+Z)+z)$ into $(x+y)+z$ if we decide that the normalisaiton form will be the complete left associative form (and every addition between an element $a$ and zero should be simplified to $a$). It will then be possible to decide the equality by simply comparing the normalised left and right hand sides with a simple syntactical equality. This is in fact what we do all the time that we have to decide if two things, written differently, are equal or not. When a human is given two mathematical polynomials and has to decide the equality of these functions, a technique that always works is to decide once and for all a normal representation of polynomials, and to put both ponynomials in this form. If the normalised form are the same, then the two original polynomials are equal, otherwise they do not represent the same computation.\\
\\
In fact, such a normalisation function can't be written directly, because in the LHS and RHS of $x=y$, we potentially have variables which have been universally quantified. And the normalization function needs to do different treatments for a "variable natural number" (a number which has been universally quantified) and for the constant $Z$. This is not possible yet, because once the variables are abstracted, they are just normal values of type $Nat$, and nothing tells us where they are in the left and right hand side of the equality. Indeed, this information only exists at the level of the ASTs representing the two terms.
For this reason, we will work by reflection. In this little example, it means that we will define a datatype which will be used as an encoding of natural numbers, or more precisely, an encoding of natural numbers composed of "variable numbers", $Z$, and additions of these things. This datatype will allow us to know the internal structure of a number, ie, where the variables and constants are.
Indeed, this datatype will allow pattern matching. Previously, we were only able to pattern match a natural number against the constructors $Z$ and $S$, which wasn't what we needed.  With the first approximation of the datatype $Expr$ presented in the Fig 6, we will be allowed to pattern match an encoding of number against the constructor $Plus$, $Var$ and $Zero$, which gives us the informations we want.


\begin{figure}[H]
\figrule
\begin{center}
\begin{verbatim}
data Expr : (gam : Vect n Nat) -> Type where
     Plus : {n:Nat} -> {gam:Vect n Nat} -> 
            Expr gam -> Expr gam -> Expr gam
     Var  : {n:Nat} -> {gam:Vect n Nat} -> 
            (i : Fin n) -> Expr gam
     Zero : {n:Nat} -> (gam:Vect n Nat) -> 
            Expr gam Z
\end{verbatim}
\end{center}
\caption{First version of reflected natural numbers}
\figrule
\end{figure}


Variables are represent using a De Brujin-like index : (Var fZ) denotes a variable, (Var (fS fZ)) another one, and so on.

The type Expr is indexed over a vector of numbers $\Gamma$, which is the context of all universally quantified variables. In the example 1, we will encode $(x1 + x2) + (x3 + x4)$ and $(x1 + (x2 + x3)) + x4$ in a context where four elements are present. The first element of this context denotes the variable $x1$, the second denotes $x2$, and so on.
Thus, the left hand side will be encoded by :

\begin{figure}[H]
\figrule
\begin{center}
\begin{verbatim}
Plus (Plus (Var gam fZ) 
           (Var gam (fS fZ))) 
     (Plus (Var gam (fS (fS fZ))) 
           (Var gam (fS (fS (fS fZ)))))
\end{verbatim}
\end{center}
\caption{Reflected LHS of example 1}
\figrule
\end{figure}

\subsection{Type safe reflection}

If we continue with this first definition of $Expr$, the normalisation function will certainly take an Expr and produce another Expr, and we will need to prove the following lemma afterwards : \\
$\forall\ e:Expr\ \Gamma,\ reify\ (reduce\ e)\ =\ reify\ e$ \\
where $reify$ is a function computing the interpretation of an Expr in a context $\Gamma$, that is to say, the natural number that this Expr is encoding.
This proof can be quite tricky to make because it relies on the complete behaviour of the reduction and on the way the interpretation is computed.
For this little example, the reduction procedure will not be too heavy, but in the next sections, with more sophisticated algebraic structures, we will have more rewriting rules to deal with and it will certainly become more problematic to "unfold" the definition of a gigantic reduction procedure. \\
\\
To avoid these two sources of complexity, we add an index to the type $Expr$, and this index is the concrete number that an Expr is encoding. Thus, it won't be necessary to define the $reify$ function, as we will know directly the concrete element reflected by a term of type $Expr$ just by looking at its index. \\


\begin{figure}[H]
\figrule
\begin{center}
\begin{verbatim}
using (x : Nat, y : Nat, gam : Vect n Nat)
  data Expr : (Vect n Nat) -> Nat -> Type where
       Plus : Expr gam x -> Expr gam y -> 
              Expr gam (x + y)
       Var  : (i : Fin n) -> Expr gam (index i gam)
       Zero : Expr gam Z
\end{verbatim}
\end{center}
\caption{Second version of reflected number with embedded denotation}
\figrule
\end{figure}

For an expression $e\ :\ Expr\ \Gamma\ x$, we will say that "$e$ denotes (or encodes) the number $x$ in the context $\Gamma$".
When an expression is a variable, the denoted number is simply the corresponding variable in the context, ie, $(index\ i\ \Gamma)$.
Also, the $Zero$ expression denotes the natural number $Z$.
Finally, if $e1$ is an expression encoding the number $x$, and $e2$ is an expression encoding the number $y$, then the expression $Plus\ e1\ e2$ denotes the number $(x + y)$.


\subsection{A correct by construction approach}

We want to write the reduction function on a "correct by construction" way, which means that no additional proof should be required after the definition of the function. Thus, $reduce$ will produce the proof that the new Expr produced has the same interpretation as the original Expr, and this will be made easier by the fact that the datatype Expr is now indexed over the real --concrete-- number : a term of type $Expr\ \Gamma\ x$ is the encoding of the number $x$.
Thus, we can write the type of $reduce$ like this : \\
$reduce\ :\ Expr\ \Gamma\ x\ \rightarrow\ (x'\ **\ (Expr\ \Gamma\ x',\ x\ =\ x'))$ \\
The function $reduce$ produces a dependent pair : the new concrete number $x'$, and a pair made of an $Expr\ \Gamma\ x'$ which is the new encoded term indexed over the new concrete number we have just produced, and a proof that old and new --concrete-- numbers are equal.
Note that this function can't simply produce an $Expr\ \Gamma\ x$, because the number on which the resulting expression will be indexed is not necessary syntactically equal to the original number since this equality can use the two available properties. Said differently, even if we can prove $x=x'$ (if the function is correctly defined), we do not have $x \equiv x'$.
And in fact, what really interest us in this function is precisely the proof of $x\ =\ x'$.
The reason is that when we try to automatically prove $x=y$, these proofs $x=x'$ and $y=y'$ will be the crucial part for the construction of the desired proof. \\
\\
We have an expression $e1$ encoding $x$, and an expression $e2$ encoding $y$\footnote{These encodings have to be produced by hand by the user for this little tool example, but for the real collection of tactics, we will program an automatic reflection mechanism}.
We will normalize $e1$, and this will give a new number $x'$, a new expression $e1':Expr\ \Gamma\ x'$, and a proof of $x=x'$. We will do the same with $e2$, and we will get a new number $y$, an expression $e2':Expr\ \Gamma\ y'$, and a proof of $y=y'$. \\
Now, we can compare $e1'$ and $e2'$ using a standard syntactical equality because these two expressions are in normal form :

\begin{figure}[H]
\figrule
\begin{center}
\begin{verbatim}
eqExpr : (e : Expr gam x) -> (e' : Expr gam y) -> Maybe (e = e')
eqExpr (Plus x y) (Plus x' y') with (eqExpr x x', eqExpr y y')
  eqExpr (Plus x y) (Plus x y) | (Just refl, Just refl) = Just refl
  eqExpr (Plus x y) (Plus x' y') | _ = Nothing
eqExpr (Var i) (Var j) with (decEq i j)
  eqExpr (Var i) (Var i) | (Yes refl) = Just refl
  eqExpr (Var i) (Var j) | _ = Nothing
eqExpr Zero Zero = Just refl
eqExpr _ _ = Nothing
\end{verbatim}
\end{center}
\caption{Syntactical equality between reflected terms}
\figrule
\end{figure}


Now, if the two normalised expressions $e1'$ and $e2'$ are equal, then they necessary have the same type\footnote{We are working with the heterogeneous equality JMeq by default in Idris, but as always, the only way to have a proof of a:A = b:B is when A=B}, and therefore $x'=y'$.
By rewriting the two equalities $x=x'$ and $y=y'$ (that we obtained during the normalisations) in the new equality $x'=y'$, we can get a proof of $x=y$.

\begin{figure}[H]
\figrule
\begin{center}
\begin{verbatim}
buildProof : {x : Nat} -> {y : Nat} -> Expr gam x' -> Expr gam y' 
           -> (x = x') -> (y = y') -> Maybe (x = y)
buildProof e e' lp rp with (eqExpr e e')
  buildProof e e lp rp | Just refl = ?MbuildProof
  buildProof e e' lp rp | Nothing = Nothing
\end{verbatim}
\end{center}
\caption{Building the desired proof with the two proofs of equality}
\figrule
\end{figure}

As we mentioned, the proof for the metavariable $MbuildProof$ is just a rewriting of the two equalities :

\begin{figure}[H]
\figrule
\begin{center}
\begin{verbatim}
  MbuildProof = proof {
  intros; refine Just; rewrite sym p1; rewrite sym p2; exact refl;
}  
\end{verbatim}
\end{center}
\caption{buildProof metavariable}
\figrule
\end{figure}

Finally, the main function which tries to prove the equality $x=y$ simply has to reduce the two metaified terms reflecting the left and the right hand side, and to use the function $buildProof$ in order to compose the two proofs we just obtained :

\begin{figure}[H]
\figrule
\begin{center}
\begin{verbatim}
  testEq : Expr gam x -> Expr gam y 
           -> Maybe (x = y)
  testEq l r = 
     let (x' ** (l', p1)) = reduce l in 
     let (y' ** (r', p2)) = reduce r in
        buildProof l' r' p1 p2 
\end{verbatim}
\end{center}
\caption{testEq}
\figrule
\end{figure}

Now, we need to define the function reduce. To do that, we have to decide a canonical representation of associative natural numbers. We decide that the left associative form will be the canonical representation. Thus, the $reduce$ function has to rewrite the metaified term by rearranging the parentheses in order to transform the underlying number in the form $(...((x1 + x2) + x3) ... + xn)$. To do so, one possibility is to define a new datatype which captures this property, and to write a function going from $Expr$ to this new type. Thus it will be easier to be certain that we are effectively computing the normal form : forcing properties to hold by the shape of a datatype is a good usage of dependent types when, like here, it doesn't introduce more complications.

\begin{figure}[H]
\figrule
\begin{center}
\begin{verbatim}
data LExpr : (gam : Vect n Nat) -> Nat -> Type where
     LPlus : LExpr gam x -> (i : Fin n) 
             -> LExpr gam (x + index i gam)
     LZero : LExpr gam Z
\end{verbatim}
\end{center}
\caption{Reflected left associative numbers}
\figrule
\end{figure}

This datatype has only two constructors. In fact, it combines the previous $Var$ and $Plus$ constructors so that it becomes impossible to write an expression which isn't left associative.
 
As part of the normalization, we write a function $expr\_l$ which converts an $Expr\ \Gamma\ x$ to a $LExpr\ \Gamma\ x'$ and which produces a proof of $x=x'$. This function will therefore use the two available properties multiple times --and especially the property of associativity--, in order to obtain the expected fully left associative form. 

\begin{figure}[H]
\figrule
\begin{center}
\begin{verbatim}
expr_l : Expr gam x 
         -> (x' ** (LExpr gam x', x = x'))
expr_l Zero = (_ ** (LZero, refl))
expr_l (Var i) = (_ ** (LPlus LZero i, refl))
expr_l (Plus ex ey) = 
  let (xl ** (xr, xprf)) = expr_l ex in
  let (yl ** (yr, yprf)) = expr_l ey in
    plusLExpr _ _ xr yr xprf yprf
      where 
      plusLExpr : (x', y' : Nat)
            -> {gam : Vect n Nat} -> {x, y : Nat} 
            -> LExpr gam x -> LExpr gam y 
            -> (x' = x) -> (y' = y) 
            -> (w' ** (LExpr gam w', x'+y'=w'))
      plusLExpr x' y' rx (LPlus e i) xprf yprf =
        let (xrec ** (rec, prf)) = 
          plusLExpr _ _ rx e refl refl in
          (_ ** (LPlus rec i, ?plusLExpr1))
      plusLExpr x' y' rx LZero xprf yprf =
        (_ ** (rx, ?plusLExpr2))
\end{verbatim}
\end{center}
\caption{Production of the left associative form}
\figrule
\end{figure}


Using this new datatype $LExpr$ has changed the representation of our encoded lists, so we need to convert back an $LExpr\ \Gamma\ x$ to an $Expr\ \Gamma\ x$. The function $l\_expr$ does this easy task.
\begin{figure}[H]
\figrule
\begin{center}
\begin{verbatim}
l_expr : LExpr gam x -> Expr gam x
l_expr LZero = Zero
l_expr (LPlus x i) = Plus (l_expr x) (Var i)
\end{verbatim}
\end{center}
\caption{Going back from LExpr to Expr}
\figrule
\end{figure}


We notice that for transforming the expression into its left associative equivalent representation, we've effectively needed to know where the variables and the $Z$ constants are : the functions $expr\_l$ and $l\_expr$ are doing different treatments for these different possibilities. \\
\\
We've got two metavariables to prove. The metavariable $plusLExpr1$ requires us to prove the goal : $x' + y' = xrec + index\ i\ \Gamma$ in a context where we've got, amongst other things,  $(xprf\ :\ x'\ =\ x)$, $(yprf\ :\ y'\ =\ x1\ +\ index\ i\ \Gamma)$ and $(prf\ :\ x\ +\ x1\ =\ xrec)$.
Proving this goal uses the property of associativity after rewriting the goal with these three proof of equality $xprf$, $yprf$ and $prf$.

\begin{figure}[H]
\figrule
\begin{center}
\begin{verbatim}
plusLExpr1 = proof {
  intros; rewrite sym xprf; rewrite sym yprf; rewrite prf;
  mrefine plusAssociative;
}
\end{verbatim}
\end{center}
\caption{Proof of the metavariable plusLExpr1}
\figrule
\end{figure}

And the metavariable plusLExpr2 uses the fact that $Z$ is a neutral element for the addition.

\begin{figure}[H]
\figrule
\begin{center}
\begin{verbatim}
plusLExpr2 = proof
  intros; rewrite xprf; rewrite (sym yprf); mrefine plusZeroRightNeutral
\end{verbatim}
\end{center}
\caption{Proof of the metavariable plusLExpr2}
\figrule
\end{figure}

We can now define the reduction, which is just the composition of the two previous functions $expr\_l$ and $l\_expr$:

\begin{figure}[H]
\figrule
\begin{center}
\begin{verbatim}
  reduce : Expr gam x -> (x' ** (Expr gam x', x = x'))
  reduce e = 
     let (x' ** (e', prf)) = expr_l e in
         (x' ** (l_expr e', prf))
\end{verbatim}
\end{center}
\caption{Reduction function}
\figrule
\end{figure}


At the moment, what we've got is not exactly a real tactic, in the sense that we only have a function which produces a value of type $Maybe (x = y)$. A real tactic would be a wrapper of this function that could properly fail with an error message when the two terms are not equal. However, here, when $x\ne y$, the function $testEq$ will simply produce the value $Nothing$. \\

\subsection{Usage of the "tactic"}

It's now time to see how to use this minimalist "tactic".
Let's define two expressions $e1$ and $e2$, respectively representing the numbers $((x + y) + (x + z))$ and $(x + ((y + x) + z))$ in the context $[x, y, z]$ of three abstracted variables.


\begin{figure}[H]
\figrule
\begin{center}
\begin{verbatim}
e1 : (x, y, z : Nat) 
    -> Expr [x, y, z] ((x+y) + (x+z))
e1 x y z = Plus (Plus (Var fZ) 
                      (Var (fS fZ))) 
                (Plus (Var fZ) 
                      (Var (fS (fS fZ))))

e2 : (x, y, z : Nat) 
     -> Expr [x, y, z] (x + ((y + x) + z))
e2 x y z = Plus (Var fZ) 
                (Plus (Plus (Var (fS fZ)) 
                            (Var fZ)) 
                      (Var (fS (fS fZ))))
\end{verbatim}
\end{center}
\caption{Two test expressions}
\figrule
\end{figure}

The numbers denoted by the expressions $e1$ and $e2$ are equal, and we can generate a proof of this by using $testEq$.

\begin{figure}[H]
\figrule
\begin{center}
\begin{verbatim}
e1_e2_testEq : (x, y, z : Nat) 
             -> Maybe (((x + y) + (x + z)) = (x + ((y + x) + z)))
e1_e2_testEq x y z = testEq (e1 x y z) (e2 x y z)
\end{verbatim}
\end{center}
\caption{Test of equality betwen e1 and e2}
\figrule
\end{figure}


And if we ask for the evaluation of this term, we should obtain $Just$ and a proof of equality between the two underlying numbers.

\begin{figure}[H]
\figrule
\begin{center}
\begin{verbatim}
#\x => \y => \z => e1_e2_testEq x y z

\x => \y => \z => Just (replace (sym (replace (sym (replace 
(plusZeroRightNeutral x) (plusAssociative x 0 y))) (replace 
(sym (replace (plusZeroRightNeutral x) (plusAssociative x 0 z))) (replace 
(replace (plusZeroRightNeutral (plus xy)) (plusAssociative (plus x y) 0 x)) 
[...]
: (x : Nat) -> (y : Nat) -> (z : Nat) 
  -> Maybe ((x + y) + x + z 
            = x+ (y + x) + z)
\end{verbatim}
\end{center}
\caption{Obtained proof}
\figrule
\end{figure}

And we effectively get the proof of equality we wanted. As expected, this proof uses the properties of associativity ($plusAssociative$) and the property of neutrality of $Z$ for $plus$ ($plusZeroRightNeutral$).


\section {Back to the general problem : A hierarchy of tactics}

Now that we have introduced our ideas on this little example, it is time to apply them for our main goal, more general : the implementation of a hierarchy of tactics proving equalities in algebraic structures. Very often, the properties available on a given type are the one of a well known algebraic structure : magma, semi-group, monoid, group, commutative-group...  We will no longer only work with the properties of associativity and of right neutral element for natural numbers as we did in the previous section, but we will have available the properties of different structures, and these tactics will usable for any type that satisfies these properties.

We will construct a hierarchy of type classes and will write one tactic for each of these type class. All our tactics (the group prover, the monoid prover...) will be able to work on any type, being given that an instance of the corresponding type class is provided.

\subsection {Hierarchy of type classes}

All our tactics will require to have a way of testing the equality between elements of the underlying set, that is to say, a way to test equality between constants. For this reason, we define a notion of "set", which only requires this equality. All the algebraic structures above will extend this type class :

\begin{figure}[H]
\figrule
\begin{center}
\begin{verbatim}
class Set c where
    set_eq : (x:c) -> (y:c) -> Maybe (x=y)
\end{verbatim}
\end{center}
\caption{Set}
\figrule
\end{figure}

Note that this decidable equality is weak in the sense that it only produces a proof when the two elements are equal, but it doesn't produce a proof of dis-equality when they are different -- instead, it simply produces the value $Nothing$--. That's quite natural, since we want to generate proof of equality, and not to generate counter examples for proving dis-equalities, which is another problem.

Obvisouly, there will be no tactic associated to $Set$, since we have no operations and no properties associated to this structure. Therefore, equalities in a $Set$ are just the syntactical equality, and they can simply be proven with $Refl$.

The first real structure, almost trivial, is the magma. A magma has just a $Plus$ operation which let us computing constants, and has no axioms about this operation.

\begin{figure}[H]
\figrule
\begin{center}
\begin{verbatim}
class Set c => Magma c where
    Plus : c -> c -> c
\end{verbatim}
\end{center}
\caption{Magma}
\figrule
\end{figure}

This code means that a type $c$ (for $carrier$) is a Magma if it is already a Set (ie, it is equiped with a $set\_eq$ operation), and if it has a $Plus$ operation.

A bit more interesting is the Semi-Group type class. A semi-Group is a Magma (ie, it still has a $Plus$ operation), but moreover it has the property of associativity for this operation.

\begin{figure}[H]
\figrule
\begin{center}
\begin{verbatim}
class Magma c => SemiGroup c where
    Plus_assoc : (c1:c) -> (c2:c) -> (c3:c) 
               -> (Plus (Plus c1 c2) c3 = Plus c1 (Plus c2 c3))
\end{verbatim}
\end{center}
\caption{Semi-Group}
\figrule
\end{figure}

And the Monoid structure is a Semi-Group with the property of neutral element.

\begin{figure}[H]
\figrule
\begin{center}
\begin{verbatim}
class SemiGroup c => Monoid c where
    Zero : c    
    Plus_neutral_1 : (c1:c) -> (Plus Zero c1 = c1)    
    Plus_neutral_2 : (c1:c) -> (Plus c1 Zero = c1)
\end{verbatim}
\end{center}
\caption{Monoid}
\figrule
\end{figure}

A Group is a Monoid, with two new operations. The binary operation $Minus$, and the unary operation $Neg$. We must have the property that $Minus$ can always be simplified with the $Neg$ and the $Plus$. That means that $Minus$ is not a "primitive" operation of a Group, since we can always rewrite $(a-b)$ into $(a\ +\ -b)$. We let the possibility to express a $Minus$ just for convenience for the user, as he won't have to rewrite by hand his $Minus$ operations with sums of negations.
The second axiom, which is the most important one for a Group, is the fact that any value $c1$ admits $Neg\ c1$ for inverse, where being an inverse is the following property :

\begin{figure}[H]
\figrule
\begin{center}
\begin{verbatim}
-- This is just a conjunctive predicate
hasSymmetric : (c:Type) -> (p:Monoid c) -> c -> c -> Type
hasSymmetric c p a b = (Plus a b = Zero, Plus b a = Zero)    
  
class Monoid c => Group c where
    Minus : c -> c -> c
    Neg : c -> c
    Minus_simpl : (c1:c) -> (c2:c) -> Minus c1 c2 = Plus c1 (Neg c2) 
    Plus_inverse : (c1:c) -> hasSymmetric c _ c1 (Neg c1)
\end{verbatim}
\end{center}
\caption{Group}
\figrule
\end{figure}


This hierarchy can be extended without difficulty with Abelian groups (with the axiom of commutativity), Ring, Commutative Rings...

These type classes will somehow be used as predicates on types. In order to call a prover on a type $c$, the user of the system will have to satisfy the corresponding type class by providing an instance of it for his type $c$, ie, he will have to prove that the corresponding properties effectively hold for $c$.

Note that the properties will either be obtained "by implementation" if the operations are real --computable-- functions, or by axioms if the user is working in an axiomatised theory where the operations ($Plus$, $Neg$, ...) are defined as axioms.

As discussed in the section 2.1 on the smaller problem, the algorithm of normalization will not directly use the concrete value of type $c$, but a reflected term, indexed over the concrete value. That still holds here.


	\subsection {Reflected terms}

We need to define a datatype for reflecting terms in each algebraic structure.
Each of these datatype is parametrised over a type $c$, which is the real type on which we want to prove equalities (the $carrier$ type). It is also indexed over an instance of the corresponding type class for $c$ (we usually call it $p$, because it behaves as a $proof$ telling that the structure $c$ has the desired properties), and indexed over a context (a vector $\Gamma$ of $n$ elements of type $c$), and also indexed over a value of type $c$, which is precisely the concrete value being encoded.
A magma is only equipped with one operation $Plus$. Thus, we only have three concepts to express in order to reflect terms in a Magma : constants, variables, and additions.



\begin{figure}[H]
\figrule
\begin{center}
\begin{verbatim}
data ExprMa : Magma c -> (Vect n c) -> c -> Type where
    ConstMa : (p : Magma c) -> (gam:Vect n c) -> (c1:c)  -> ExprMa p gam c1 
    PlusMa : {p : Magma c} -> {gam:Vect n c} -> {c1:c} -> {c2:c} 
         -> ExprMa p gam c1 -> ExprMa p gam c2 
         -> ExprMa p gam (Plus c1 c2) 
    VarMa : (p:Magma c) -> {gam:Vect n c}
         -> (i:Fin n) -> ExprMa p gam (index i gam)
\end{verbatim}
\end{center}
\caption{Reflected terms for a Magma}
\figrule
\end{figure}

When we encode a constant $c1$ in a context $\Gamma$, we use the constructor $ConstMa$ to produce a term of type $ExprMa\ p\ \Gamma\ c1$ : the index representing the concrete value is precisely this constant $c1$.
If e1 is an expression of type $ExprMa\ p\ \Gamma\ c1$ (ie, a term encoding the value $c1$), and $e2$ is an expression of type $ExprMa\ p\ \Gamma\ c2$ (ie, a term encoding the value $c2$), then the term $PlusMa\ e1\ e2$ will have the type $ExprMa\ p\ \Gamma\ (Plus\ c1\ c2)$, ie, this term will encode the value $(Plus\ c1\ c2)$, where $Plus$ is the operation defined in the current instance $p$.


%For the moment, the definition of $Variable$ is just the following:
%
%\begin{code}[caption=Reflected variables, captionpos=b, label=lst1:haskell2]  
%data Variable : {c:Type} -> {n:Nat}
%  -> (c_equal : (c1:c)->(c2:c)->Maybe(c1=c2)) 
%  -> (Vect n c) -> c -> Type where
%    RealVariable : (c_equal:(c1:c)->(c2:c)
%                    ->Maybe(c1=c2))
%          -> (gam:Vect n c) -> (i:Fin n) 
%          -> Variable c_equal gam (index i gam) 
%\end{code}	
%
%This type will be extended with some other constructors later on. For the moment, we can only %express a "real variable", referred by its index $i$. \\
%\\
Because there is no additional operations in a SemiGroup or in a Monoid, the datatypes for reflected terms in these two structures will have exactly the same shape as the one for Magma that we've given above.
However, the one for $Group$ will introduce two new constructors for the $Neg$ and $Minus$ operations :


\begin{figure}[H]
\figrule
\begin{center}
\begin{verbatim}
data ExprG :  Group c -> (Vect n c) -> c -> Type where
    ConstG : (p : dataTypes.Group c) -> (gam:Vect n c) 
      -> (c1:c) -> ExprG p gam c1
    PlusG : {p : Group c} -> {gam:Vect n c} -> {c1:c} -> {c2:c} 
      -> ExprG p gam c1 -> ExprG p gam c2 
      -> ExprG p gam (Plus c1 c2)
    MinusG : {p : Group c} -> {gam:Vect n c} -> {c1:c} -> {c2:c} 
      -> ExprG p gam c1 -> ExprG p gam c2 
      -> ExprG p gam (Minus c1 c2)
    NegG : {p : Group c} -> {gam:Vect n c} -> {c1:c} 
      -> ExprG p gam c1 -> ExprG p gam (Neg c1)
    VarG : (p : Group c) -> {gam:Vect n c} 
      -> (i:Fin n) -> ExprG p gam c1
\end{verbatim}
\end{center}
\caption{Reflected terms in a Group}
\figrule
\end{figure}

The index of type $c$ (the real value represented by an expression) is always expressed by using the available operations in the type class $p$, which for a group Group are $Plus$, $Minus$ and $Neg$.

	\subsection {A bit of notation}
The equality we are trying to prove is $x=y$, where $x$ and $y$ are elements of the type $c$, which  simulates a set with some properties (which make $c$ being a Semi-Group, or a Monoid, or a Group...). The fact that $c$ fulfils the specification of an algebraic structure is expressed as an instance of the corresponding type class, and this instance will be denoted as $p$.
The reflected term for $x$ will be denoted $e1$, and this term will have the type $ExprG\ p\ \Gamma\ x$. The term $e1$ is the encoding of $x$ and its type is precisely indexed over the real value $x$, as the reflected terms embed the concrete values.
We've got similar thing for $y$, which is encoded by $e2$, and its type is indexed over the real value $y$.
Running the normalisation procedure on $e1$ will produce the normal form $e1'$ of type $ExprG\ p\ \Gamma\ x'$ and a proof $p1$ of $x=x'$ while running the normalisation procedure on $e2$ will produce the normal form $e2'$ of type $ExprG\ p\ \Gamma\ y'$ and a proof $p2$ of $y=y'$.

	\subsection {Deciding equality}
	
Each algebraic structure will have a function for reducing the reflected terms into their normal forms. The fact that all of these algebraic structures admit a canonical represent for any element is in fact a very nice property that we are using in order to decide equalities. Without this property, it would become a lot more complicated to decide the equality without bruteforcing a serie of rewriting, that might even not terminate.
These functions will compute this canonical represent, by rewriting the given term multiple times, and at the same time, it will produce the proof of equality between the real value indexing the original term and the real value indexing the produced term.

\begin{figure}[H]
\figrule
\begin{center}
\begin{verbatim}
	groupReduce : {c:Type} -> {n:Nat} -> (p:Group c) -> {gam:Vect n c} 
	  -> {x:c} -> (ExprG p gam x) -> (x' ** (ExprG p gam x', x=x'))
\end{verbatim}
\end{center}
\caption{Type of the reduction function for terms reflecting elements in a Group}
\figrule
\end{figure}

This function has more work to do in structures with multiple axioms (like Group), than for the easier structure underneath.
The details of these functions will be given in the next section. For the moment, we just assume that we've got such a function for each structure. We continue to give here the details for the Group structure, but the principle is exactly the same with the other structures, and the exact equivalent of what we did in the previous section with natural numbers and additions.

We want to write the following function :

\begin{figure}[H]
\figrule
\begin{center}
\begin{verbatim}
	groupDecideEq : (p:Group c) -> {gam:Vect n c} -> {x : c} -> {y : c} 
	   -> (ExprG p gam x) -> (ExprG p gam y) 
	   -> Maybe (x = y)
\end{verbatim}
\end{center}
\caption{Type of the function for deciding equality between elements in a Group}
\figrule
\end{figure}

The first thing this function has to do is to compute the normal form of the two expressions $e1$ and $e2$ respectively encoding $x$ and $y$.
Then, the only thing remaining to do will be to compare syntactically $e1'$ and $e2'$, and if they are equal, then we will get $x'=y'$, and we will be able to produce the desired proof of $x=y$ by using the two proofs $p1$ and $p2$. 

\begin{figure}[H]
\figrule
\begin{center}
\begin{verbatim}
groupDecideEq p e1 e2 =
  let (x' ** (e1', p1)) = groupReduce p e1 in
  let (y' ** (e2', p2)) = groupReduce p e2 in
	     buildProofGroup p e1' e2' p1 p2
\end{verbatim}
\end{center}
\caption{Decides if two elements of a group are equal}
\figrule
\end{figure}

The syntactical test of equality between $e1'$ and $e2'$ and the composition of the two proofs is done in the auxiliary function $buildProofGroup$, similarly to what we've done on the smaller example in the previous section. 

\begin{figure}[H]
\figrule
\begin{center}
\begin{verbatim}
buildProofGroup : (p:Group c) -> {gam:Vect n c} 
  -> {x : c} -> {y : c} -> {x':c} -> {y':c} 
  -> (ExprG p gam x') -> (ExprG p gam y') 
  -> (x = x') -> (y = y') 
  -> (Maybe (x = y))
buildProofGroup p e1 e2 p1 p2 with (exprG_eq p _ e1 e2)
    buildProofGroup p e1 e1 p1 p2 | Just refl = ?MbuildProofGroup
    buildProofGroup p e1 e2 p1 p2 | Nothing = Nothing
\end{verbatim}
\end{center}
\caption{Composes the two proofs if the normal forms are the same}
\figrule
\end{figure}

The proof for the metavariable $MbuildProofGroup$ is exactly the same as the corresponding one (called $MbuildProof$) in the smallest example showed in section 2.3.


\subsection {Automatic reflection}
$reflect : \{n:Nat\} \rightarrow (gamma : Vect\ n\ c) \rightarrow (c1:c) \rightarrow Expr\ gamma\ c1$.
The reflected term is guaranteed to be a faithful representation of the input $c1$. We can write this function because Idris has a $\%syntax$ command which let us do pattern maching on syntax (ie, it let us access the AST of a term), and this $\%syntax$ produces functions which are typed ! (unlike Coq's $Ltac$). 












\section {Normalisation functions and re-usability of the provers}

We now describe briefly how the normalisation functions work for some of the structures, and we also explain how we've been able to reuse each prover for the more specialised underneath structures. 

	\subsection {Normalization of terms in a Semi-Group}
In a semigroup, we only have to deal with the property of associativity. As it was the case with the toy example in the previous section, we will have to rearrange the parenthesis on a systematical way, either left associative, or right associative. But this is not the only thing that we have to do. If $x$, $y$ and $z$ denote variables, we don't only want to transform $x+((y+4)+(5+z))$ into $(((x+y)+4)+5)+z$ if we've chosen the complete left associative form. We want first to do the addition of the constants $4$ and $5$ together (and we have the right to do so because we can rearrange the parenthesis thanks to the associativity, when constants are, like in this case, near to each other). Three possible patterns need this treatment : $(x+c1)+(c2+y)$ , $(x + c1) + c2$  and $c1 + (c2 + x)$ when $c1$ and $c2$ denote constants. The simplification between the constants $c1$ and $c2$ is denoted here as $[c1+c2]$, and this computation is done by the normalisation function of the underneath $Magma$ level, called $magmaReduce$.
The first pattern is rewritten into $(x + [c1+c2]) + y$, the second into $x + [c1+c2]$ and the third one into $[c1 + c2] + x$.
We show the code for the first pattern.

\begin{figure}[H]
\figrule
\begin{center}
\begin{verbatim}
assoc : (p:SemiGroup c) -> (gam:Vect n c) -> {c1:c} 
        -> (ExprSG p gam c1) -> (c2 ** (ExprSG p gam c2, c1=c2))
assoc p gam (PlusSG 
             (PlusSG e1 (ConstSG _ _ c1)) 
             (PlusSG (ConstSG _ _ c2) e2)) =
    let (r_ih1 ** (e_ih1, p_ih1)) = (assoc p gam e1) in
    let (r_ih2 ** (e_ih2, p_ih2)) = (assoc p gam e2) in
    let (r_3 ** (e_3, p_3)) = magmaReduce (semiGroup_to_magma 
                             (PlusSG (ConstSG _ _ c1) (ConstSG _ _ c2))) in
    let e_3' = magma_to_semiGroup p e_3 in
        (_ ** ((PlusSG (PlusSG e_ih1 e_3') e_ih2), ?Massoc1))
[...]
\end{verbatim}
\end{center}
\caption{Computing with associativity in a Semi-Group, first pattern}
\figrule
\end{figure}

Note that we use function of conversions between the different levels : $semiGroup\_to\_magma$ and $magma\_to\_semiGroup$.

After simplifying the constants computable thanks to associativity, the normalization for SemiGroup also has to rearrange the brackets on a systematic way. The fully right or fully left associative forms can both be chosen for this purpose, and the implementation is very similar to the function $expr_l$ previously described on section 2.3.

\subsection {Normalization of terms in a Monoid}

In a monoid, we simply have to eliminate zeros, thanks to the two properties of left and right neutrality. Because the simplification of additions between a zero and a constant has be done at the underneath level of semigroup (by computing the addition of these two constants), we only have to simplify the addition between the constant zero and a variable. There is in fact two cases, since the variable can come first, or the $Zero$ can come first. We can write a function $elimZero$ which rewrites these two patterns $x+Z$ and $Z+x$ as $x$.

%\begin{figure}[H]
%\figrule
%\begin{center}
%\begin{verbatim}
%elimZero : (p:Monoid c)->(gam:Vect n c)->{c1:c} 
%            -> (ExprMo p gam c1) 
%            -> (c2 ** (ExprMo p gam c2, c1=c2))
%elimZero c p gam (PlusMo (ConstMo _ _ const1) (VarMo _ v)) 
%        with (eq_on_set p Zero const1)
%   elimZero c p gam (PlusMo (ConstMo _ _ Zero) (VarMo _ v)) 
%        | (Just refl) = (_ ** (VarMo _ v, ?MelimZero1))
%   elimZero c p gam (PlusMo (ConstMo _ _ const1) (VarMo _ v)) 
%        | _ = (_ ** (PlusMo (ConstMo _ _ const1) (VarMo _ v), refl)) 
%elimZero c p gam 
% (PlusMo (VarMo _ v) (ConstMo _ _ const2)) 
%      with (eq_as_elem_of_set p Zero const2) 
%   elimZero c p gam (PlusMo (VarMo _ v) 
%            (ConstMo _ _ Zero)) | (Just refl) 
%          = (_ ** (VarMo _ _ v, ?MelimZero2))
%   elimZero c p gam (PlusMo (VarMo _ v) 
%            (ConstMo _ _ const2)) | _ 
%          = (_ ** (PlusMo (VarMo _ v) 
%                 (ConstMo _ _ const2), refl))
%[...]
%\end{verbatim}
%\end{center}
%\caption{Eliminating zeros in a Monoid, first pattern}
%\figrule
%\end{figure}

The normalization of reflected expressions in a monoid is simply made of a call to the function reducing terms on a semigroup, followed by a call to elimZero which will eliminates the remaining additions with zeros. 


\subsection {Reusing the monoid prover for building the group prover}

The first thing that the reduction function of the group prover will have to do is simply to transform every subtraction $a-b$ into $a + -b$, and we are entitled to do so precisely because we have available the property $Minus\_simpl$ contained in the instance of the type class.
After that, the reduction continues with the propagation of the $Neg$ operation inside the parenthesis and $-(a+b)$ is transformed into $(-b) + (-a)$. Note that we have to be careful and not simplify it to $(-a) + (-b)$ as it would assume that we're having a commutative monoid.
Then, because we've pushed some negations inside the parenthesis, we might have sequences of two or more consecutive negations. We simplify them by removing two consecutive negations every time that we find such a sequence. Once this is done, there is a last major step specific of groups to accomplish, which is the simplification of sums of symmetric elements. 

This might be a direct sum of symmetric elements, like $-e1 + e2$ or $e1 + -e2$ and in these two cases, we need to simplify them when $e1$ can be reduced\footnote{It is not enough to simply check their syntactical equality, as they might not be immediately equal but could perhaps be normalised to the same term. Thus, this step --which is part of the reduction function-- uses the entire reduction function, but it calls it on a smaller term} to $e2$ , and if so the result should be $Zero$. We are entitled to do this treatment because of the property $Plus\_inverse$.

The sum to simplify might be more complex, with two or three levels of of $Plus$. In these cases, there might be some simplifications to do thanks to the associativity property. With two levels of sums, we will transform $(e1 + (-e2 + e3))$ and $(-e1 + (e2+e3))$ into $e3$ when $e1$ can be normalised to $e2$. Also, $((e1+e2) + -e3))$ and $((e1+(-e2)) + e3)$ are simplified into $e1$ when $e2$ can be reduced to $e3$. Finally, in presence of three level of sums, we will transform $(a+b) + ((-c)+d)$ and $(a+(-b)) + (c+d)$ into $a+d$ when $b$ can be reduced to $c$.


The idea : negations $-x$ are encoded as variables.

\subsection {Reusing the commutative group prover for building the ring prover}

The idea : monomials (containing products), like $3*(x*y)$ are encoded as variables when passed to the commutative group prover. 



\section {Discussion}

	\subsection{Comparison with a traditional approach}

What we would do without type-safe reflection, and when the normalisation functions are not correct by construction : defining a (weekly typed) normalisation a function, and latter on, a lemma about this normalisation function. Building the tactic with this lemma.
cf extra slides 33-34-35-36 of my STP talk.

	\subsection {Comparison with Coq's approach}
	
Exactly as above, but let the typechecker do more work.
cf extra slides 37-38 of my STP talk.

	\subsection {Correctness and completeness}

		\subsubsection{Correctness}
		
Idea : obtained by construction. In fact, what we really needed to produce was the proof of equality, and not the normalised terms. The only two things we have to ensure are :
\begin{itemize}
\item not having introduces any axioms
\item not having introduced any non total function (is it really needed in fact if all this stuff only exists at typecheck and not at runtime? I guess it's fine as long as we haven't used non-total functions to illegitimately extract proof of lemmas)
\end{itemize}

If we meet the two conditions above, we simply can't produce a false proof of a statement, because the typechecker would complain. Like in Coq's approach where they just see if the typecheck is ok to apply $f\_correct\ (reflect\ a)\ (reflect\ b)$ in order to prove $a=b$. The typecheck only agrees if $reify\ (reflect\ a) = reify(reflect\ b)$ can be unified with $a=b$, which is the case if the $reify$ and $reflect$ functions have been correctly defined.
		
		\subsubsection{Completeness}
		
Idea 1 : Having a complete tactic is important, because otherwise the tactic is not usable often, and thus is not really valuable. However, having a formal proof of the completeness in the system itself isn't really interesting. As long as we do not meet an equality that should be provable, and which isn't proved by our tactic, there is no problem. And if this situation does happen, nothing really bad really happened : no inconsistencies have been introduced, etc. The tactic just refuses to do it automatically for the programmer. The human can still try to attempt it. \\
\\
Idea 2 : Completeness is not, as one could imagine : \\
$a=b \rightarrow norm\ (reflect\ a) = norm\ (reflect\ b)$
The reason is that we could just rewrite the hypothesis in the goal, without having to inspect the behaviour of the $norm$ function. That would prove nothing interesting.
Completeness is in fact $norm\ (reflect\ a) \neq norm(reflect\ b) \rightarrow a \neq b$, which is in fact :
$decideEq\ (reflect\ a)\ (reflect\ b) = Nothing \rightarrow a \neq b$.
This thing is terrible to prove in the proof assistant. It has never been done for this problem (not for Coq's ring prover, not for Agda ring prover). Can I do it for the little example with natural numbers though ? \\
\\
Idea 3 : We have the guarantee that we aren't losing any totality because of the $reflect$ function because of our index ! This is another extra advantage of our approach. We know that $reify\ (reflect\ a)$ is always unifiable with $a$ in our case. Coq's approach doesn't have this guarantee, and could potentially lose some completeness if this property is not always true.


\begin{thebibliography}{}
 \bibitem[\protect\citename{B. Gregoire and A. Mahboubi}2005]{Coq2005}
   Gregoire,~B. and Mahboubi,~A. (2005) Proving equalities in a commutative ring done right in Coq, TPHOLs 2005, Oxford, UK, August 22-25, 2005, Proceedings


 \bibitem[\protect\citename{W.A Howard}1980]{How80}
   Howard W.A. The formulae-as-types notion of construction. In J. R. Seldin and J. P. Hindley, editors, To H. B. Curry: Essays on Combinaory Logic, Lambda Calculus, and Formalism. Academic Press, 1980.


\end{thebibliography}

\label{lastpage}

\end{document}

% end of JFP2egui.tex
